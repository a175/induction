% !TeX root =./x2.tex
% !TeX program = pdfpLaTeX
\section{記号}
$\NN$は非負整数全体のなす集合,
$\ZZ_{>0}$は正の実数全体のなす集合.

$n!$は階乗, つまり, $n\in \ZZ_{>0}$に対し, $n!=\prod_{i=1}^n i$;
$0!=1$.

$n!!$は二重階乗, つまり, $k\in \ZZ_{>0}$に対し, 
$(2k-1)!!=\prod_{i=1}^k (2i-1)$, $(2k)!!=\prod_{i=1}^k (2i)$;
$0!!=1$.

$\binom{x}{n}$は二項係数, つまり,
$x\in\RR$, $n\in \NN$に対し,
$\binom{x}{n}=\frac{\prod_{i=0}^{n-1}(x-i)}{n!}$.

\section{通常の数学的帰納法によるもの}
\subsection{等式に関するもの}
\subsubsection{整数や整式の和に関するもの}
\begin{prop}
  \label{p:20230630}
  $\forall n\in\NN$,
  $\sum_{i=0}^{n}i=\frac{n(n+1)}{2}$.
\end{prop}
\begin{proof**}
  $0=\frac{0(0+1)}{2}$である.
  また,
  $n+\sum_{i=0}^{n-1}i=\sum_{i=0}^{n}i$,
  $n+\frac{(n-1)n}{2}=\frac{n(n+1)}{2}$であるので,
  $n$に関する数学的帰納法により示せる.
\end{proof**}
\begin{proof*}
  $P(n)$を次の命題とする:
  \begin{align*}
    \sum_{i=0}^{n}i=\frac{n(n+1)}{2}.
  \end{align*}
  このとき, 全ての$n\in\NN$で$P(n)$が成り立つことを,
  数学的帰納法で示す.

  \paragraph{Base Case:}
  $P(0)$が成り立つことは, 以下から明らか:
  \begin{align*}
    \sum_{i=0}^{0}i&=0,\\
    \frac{0(0+1)}{2}&=0.
  \end{align*}

  \paragraph{Induction Step:}
  $P(n-1)\implies P(n)$を示す.
  仮定から$\sum_{i=0}^{n-1}i=\frac{(n-1)n}{2}$であるので,
  \begin{align*}
    \sum_{i=0}^{n}i&=n+\sum_{i=0}^{n-1}i\\
    &=n+\frac{(n-1)n}{2}\\
    &=\frac{2n+(n-1)n}{2}\\
    &=\frac{2n+n^2-n}{2}\\
    &=\frac{n^2+n}{2}\\
    &=\frac{n(n+1)}{2}.
    \qedhere
  \end{align*}
\end{proof*}
\begin{rem}
  数学的帰納法を用いず, 以下のように示す方が一般的だと思う:
  \begin{align*}
    \sum_{i=0}^{n}i&=\frac{\sum_{i=0}^{n}i + \sum_{i=0}^{n}i}{2}\\
    &=\frac{\sum_{i=0}^{n}i + \sum_{i=0}^{n}(n-i)}{2}\\
    &=\frac{\sum_{i=0}^{n}(i + n-i)}{2}\\
    &=\frac{\sum_{i=0}^{n}n}{2}\\
    &=\frac{n\sum_{i=0}^{n}1}{2}\\
    &=\frac{n(n+1)}{2}.
  \end{align*}
\end{rem}

\begin{rem}
  数学的帰納法を用いず, 以下のように示すこともできる:
  \begin{align*}
    X&=\Set{(t_1,t_2)|0\leq t_1<t_2 \leq  n}\\
    X_i&=\Set{(t_1,i)|0\leq t_1<i }
  \end{align*}
  とおく. このとき,
  \begin{align*}
    \coprod_{i=0}^{n}X_i = X
  \end{align*}
  であり,
  \begin{align*}
    \numof{X}&=\binom{n+1}{2}=\frac{n(n+1)}{2}\\
    \numof{X_i}&=i
  \end{align*}
  であるので,
  \begin{align*}
    \sum_{i=0}^{n}i=\frac{n(n+1)}{2}.
  \end{align*}
\end{rem}

\begin{rem}
\cref{p:20230717,p:20230718}
は,
この一般化である.
\end{rem}

\begin{prop}
  \label{p:20230703}
  $\forall n\in\NN$,
  $\sum_{i=0}^{n}2i=n(n+1).$
\end{prop}
\begin{proof**}
  $2\cdot 0=0=0(0+1)$である.
  また,
  $2n+\sum_{i=0}^{n-1}2i=\sum_{i=0}^{n}2i$,
  $2n+(n-1)n=n(n+1)$であるので,
  $n$に関する数学的帰納法により示せる.
\end{proof**}


\begin{proof*}
  $P(n)$を次の命題とする:
  \begin{align*}
    \sum_{i=0}^{n}2i=n(n+1).
  \end{align*}
  このとき, 全ての$n\in\NN$で$P(n)$が成り立つことを,
  数学的帰納法で示す.

  \paragraph{Base Case:}
  $P(0)$が成り立つことは, 以下から明らか:
  \begin{align*}
    \sum_{i=0}^{0}2i&=2\cdot 0=0,\\
    0(0+1)&=0.
  \end{align*}

  \paragraph{Induction Step:}
  $P(n-1)\implies P(n)$を示す.
  仮定から$\sum_{i=0}^{n-1}2i=(n-1)n$であるので,
  \begin{align*}
    \sum_{i=0}^{n}2i&=2n+\sum_{i=0}^{n-1}2i\\
    &=2n+(n-1)n\\
    &=2n+n^2-n\\
    &=n^2+n\\
    &=n(n+1).
    \qedhere
  \end{align*}
\end{proof*}

\begin{rem}
  \cref{p:20230630}をみとめ,
  数学的帰納法を用いず, 以下のように示す方が一般的だと思う:
  \begin{align*}
    \sum_{i=0}^{n}2i&=
    2\sum_{i=0}^{n}i
    =2\frac{n(n+1)}{2}=n(n+1).
  \end{align*}
\end{rem}

\begin{prop}
  \label{p:20230704}
  $\forall n\in\ZZ_{>0}$, $\sum_{i=1}^{n}(2i-1)=n^2$.
\end{prop}

\begin{proof**}
  $2\cdot 1-1=1=1^2$である.
  また,
  $(2n-1)+\sum_{i=1}^{n-1}(2i-1)=\sum_{i=1}^{n}(2i-1)$,
  $(2n-1)+(n-1)^2=n^2$であるので,
  $n$に関する数学的帰納法により示せる.
\end{proof**}

\begin{proof*}
  $P(n)$を次の命題とする:
  \begin{align*}
    \sum_{i=1}^{n}(2i-1)=n^2.
  \end{align*}
  このとき,
  全ての$n\in\ZZ_{>0}$で$P(n)$が成り立つことを,
  数学的帰納法で示す.

  \paragraph{Base Case:}
  $P(1)$が成り立つことは, 以下から明らか:
  \begin{align*}
    \sum_{i=1}^{1}(2i-1)&=2\cdot 0-1=1,\\
    1^2&=1.
  \end{align*}

  \paragraph{Induction Step:}
  $P(n-1)\implies P(n)$を示す.
  仮定から$\sum_{i=1}^{n-1}(2i-1)=(n-1)^2$であるので,
  \begin{align*}
    \sum_{i=1}^{n}(2i-1)&=2n-1+\sum_{i=0}^{n-1}(2i-1)\\
    &=2n-1+(n-1)^2\\
    &=2n-1+n^2-2n+1\\
    &=n^2.
    \qedhere
  \end{align*}
\end{proof*}

\begin{rem}
  \cref{p:20230630}をみとめ,
  数学的帰納法を用いず, 以下のように示す方が一般的だと思う:
  \begin{align*}
    \sum_{i=1}^{n}(2i-1)&=
    2\sum_{i=1}^{n}i-\sum_{i=1}^n 1
    =2\frac{n(n+1)}{2}-n=n(n+1)-n=n^2.
  \end{align*}
\end{rem}

\begin{prop}
  \label{p:20230705}
  $\forall n\in\NN$, $\sum_{i=0}^{n}i^2=\frac{n(n+1)(2n+1)}{6}$.
\end{prop}
\begin{proof**}
  $0^2=0=\frac{0\cdot 1\cdot 1}{6}$である.
  また,
  $n^2+\sum_{i=0}^{n-1}i^2=\sum_{i=0}^{n}i^2$,
  $n^2+\frac{(n-1)n(2n-1)}{6}=\frac{2n^3-3n^2+n+6n^2}{6}=\frac{2n^3+3n^2+n}{6}=\frac{n(n+1)(2n+1)}{6}$であるので,
  $n$に関する数学的帰納法により示せる.
\end{proof**}


\begin{proof*}
  $P(n)$を次の命題とする:
  \begin{align*}
    \sum_{i=0}^{n}i^2=\frac{n(n+1)(2n+1)}{6}.
  \end{align*}
  このとき,
  全ての$n\in\NN$で$P(n)$が成り立つことを,
  数学的帰納法で示す.

  \paragraph{Base Case:}
  $P(0)$が成り立つことは, 以下から明らか:
  \begin{align*}
    \sum_{i=1}^{1}i^2&=0^2=0,\\
    \frac{0(0+1)(2\cdot 0+1)}{6}&=0.
  \end{align*}

  \paragraph{Induction Step:}
  $P(n-1)\implies P(n)$を示す.
  仮定から$\sum_{i=0}^{n-1}i^2=\frac{(n-1)n(2n-1)}{6}$であるので,
  \begin{align*}
    \sum_{i=1}^{n}i^2&=n^2+\sum_{i=0}^{n-1}i^2\\
    &=\frac{6n^2+2n^3-3n^2+n}{6}\\
    &=\frac{2n^3+3n^2+n}{6}\\
    &=\frac{n(n+1)(2n+1)}{6}
    .\qedhere
  \end{align*}
\end{proof*}

\begin{rem}
  \cref{p:20230630}をみとめ,
  数学的帰納法を用いず, 以下のように示す方が一般的だと思う:
  \begin{align*}
    S&=\sum_{i=0}^n i^2\\
    T&=\sum_{i=1}^{n}(i^3-(i-1)^3)
  \end{align*}
  とする.
  このとき,
  \begin{align*}
    T&=\sum_{i=1}^{n}i^3-\sum_{i=1}^n (i-1)^3\\
    &=\sum_{i=1}^{n}i^3-\sum_{i=0}^{n-1} i^3\\
    &=(n^3+\sum_{i=1}^{n-1}i^3)-(\sum_{i=1}^{n-1} i^3)+0^3)\\
    &=n^3.
  \end{align*}
  一方次のようにも計算できる:
  \begin{align*}
    T
    &=\sum_{i=1}^{n}(i^3- (i^3-3i^2+3i-1))\\
    &=\sum_{i=1}^{n}(3i^2-3i+1)\\
    &=3\sum_{i=1}^{n}i^2-3\sum_{i=1}^{n}i+\sum_{i=1}^{n}1\\
    &=3S-3\frac{n(n+1)}{2}+n.
  \end{align*}
  したがって,
  \begin{align*}
    n^3&=3S-3\frac{n(n+1)}{2}+n\\
    3S&=n^3+3\frac{n(n+1)}{2}-n\\
    &=\frac{2n^3+3n(n+1)-2n}{2}\\
    &=\frac{n(2n^2+3(n+1)-2)}{2}\\
    &=\frac{n(2n^2+3n+1)}{2}\\
    &=\frac{n(n+1)(2n+1)}{2}\\
    S&=\frac{n(n+1)(2n+1)}{6}.
  \end{align*}
\end{rem}

\begin{prop}
  \label{p:20230706}
  $\forall n\in\ZZ_{>0}$, $\sum_{i=1}^{n}(2i-1)^2=\frac{n(2n-1)(2n+1)}{3}$.
\end{prop}
\begin{proof**}
  $1^2=1=\frac{1\cdot 1\cdot 3}{3}$である.
  また,
  \begin{align*}
    (2n-1)^2+\sum_{i=1}^{n-1}(2i-1)^2&=\sum_{i=1}^{n}(2i-1)^2\\
    (2n-1)^2+\frac{(n-1)(2n-3)(2n-1)}{3}
    &=\frac{3(2n-1)^2+(n-1)(2n-3)(2n-1)}{3}\\
    &=\frac{(2n-1)(3(2n-1)+(n-1)(2n-3))}{3}\\
    &=\frac{(2n-1)(6n-3+n^2-5n+3))}{3}\\
    &=\frac{(2n-1)(2n^2+n)}{3}\\
    &=\frac{n(2n-1)(2n+1)}{3}
  \end{align*}
  であるので,
  $n$に関する数学的帰納法により示せる.
\end{proof**}
\begin{proof*}
  $P(n)$を次の命題とする:
  \begin{align*}
    \sum_{i=1}^{n}(2i-1)^2=\frac{n(2n-1)(2n+1)}{3}
  \end{align*}
  このとき,
  全ての$n\in\ZZ_{>0}$で$P(n)$が成り立つことを,
  数学的帰納法で示す.

  \paragraph{Base Case:}
  $P(1)$が成り立つことは, 以下から明らか:
  \begin{align*}
    \sum_{i=1}^{1}(2i-1)^2&=1^2=1,\\
    \frac{1(2\cdot 1-1)(2\cdot 1+1)}{3}&=1.
  \end{align*}

  \paragraph{Induction Step:}
  $P(n-1)\implies P(n)$を示す.
  仮定から$\sum_{i=1}^{n-1}(2i-1)^2=\frac{(n-1)(2n-3)(2n-1)}{3}$であるので,
  \begin{align*}
    \sum_{i=1}^{n}(2i-1)^2&=(2n-1)^2+\sum_{i=0}^{n-1}(2i-1)^2\\
    &=(2n-1)^2+\frac{(n-1)(2n-3)(2n-1)}{3}\\
    &=\frac{3(2n-1)^2+(n-1)(2n-3)(2n-1)}{3}\\
    &=\frac{(2n-1)(3(2n-1)+(n-1)(2n-3))}{3}\\
    &=\frac{(2n-1)((6n-3)+(2n^2-5n+3))}{3}\\
    &=\frac{(2n-1)(2n^2+n)}{3}\\
    &=\frac{n(2n-1)(2n+1)}{3}
    .\qedhere
  \end{align*}
\end{proof*}


\begin{rem}
  \cref{p:20230630,p:20230705}をみとめ,
  数学的帰納法を用いず, 以下のように示す方が一般的だと思う:
  \begin{align*}
    \sum_{i=1}^{n}(2i-1)^2&=
    4\sum_{i=1}^{n}i^2-4\sum_{i=1}^n i+\sum_{i=1}^n 1\\
    &=
    4\frac{n(n+1)(2n+1)}{6}
    -4\frac{n(n+1)}{2}+n\\
    &=\frac{4n(n+1)(2n+1)-12n(n+1)+6n}{6}\\
    &=\frac{n(4(n+1)(2n+1)-12(n+1)+6)}{6}\\
    &=\frac{n((n+1)(4(2n+1)-12)+6)}{6}\\
    &=\frac{n((n+1)(8n-8)+6)}{6}\\
    &=\frac{n(8(n+1)(n-1)+6)}{6}\\
    &=\frac{n(8(n^2-1)+6)}{6}\\
    &=\frac{n(8n^2-2)}{6}\\
    &=\frac{n(4n^2-1)}{3}\\
    &=\frac{n(2n-1)(2n+1)}{3}.
  \end{align*}
\end{rem}


\begin{prop}
  \label{p:20230707}
  $\forall n\in\NN$, $\sum_{i=0}^{n}(2i)^2=\frac{2n(n+1)(2n+1)}{3}$.
\end{prop}
\begin{proof**}
  $0^2=0=\frac{2\cdot 0\cdot 1\cdot 1}{3}$である.
  また,
    $(2n)^2+\sum_{i=1}^{n-1}(2i)^2=\sum_{i=1}^{n}(2i)^2$,
    $(2n)^2+\frac{2(n-1)n(2n-1)}{3}=\frac{2n(n+1)(2n+1)}{3}$
  であるので,
  $n$に関する数学的帰納法により示せる.
\end{proof**}


\begin{proof*}
  $P(n)$を次の命題とする:
  \begin{align*}
    \sum_{i=0}^{n}(2i)^2=\frac{2n(n+1)(2n+1)}{3}
  \end{align*}
  このとき,
  全ての$n\in\NN$で$P(n)$が成り立つことを,
  数学的帰納法で示す.

  \paragraph{Base Case:}
  $P(0)$が成り立つことは, 以下から明らか:
  \begin{align*}
    \sum_{i=0}^{0}(2i)^2&=0^2=0,\\
    \frac{2\cdot 0(0+1)(2\cdot 0+1)}{3}&=0.
  \end{align*}

  \paragraph{Induction Step:}
  $P(n-1)\implies P(n)$を示す.
  仮定から$\sum_{i=0}^{n-1}(2i)^2=\frac{2(n-1)n(2n-1)}{3}$であるので,
  \begin{align*}
    \sum_{i=1}^{n}(2i)^2&=(2n)^2+\sum_{i=0}^{n-1}(2i)^2\\
    &=(2n)^2+\frac{2(n-1)n(2n-1)}{3}\\
    &=\frac{3(2n)^2+2(n-1)n(2n-1)}{3}\\
    &=\frac{2n(6n+(n-1)(2n-1))}{3}\\
    &=\frac{2n(6n+2n^2-3n+1)}{3}\\
    &=\frac{2n(2n^2+3n+1)}{3}\\
    &=\frac{2n(n+1)(2n+1)}{3}
    .\qedhere
  \end{align*}
\end{proof*}


\begin{rem}
  \cref{p:20230705}をみとめ,
  数学的帰納法を用いず, 以下のように示す方が一般的だと思う:
  \begin{align*}
    \sum_{i=1}^{n}(2i)^2&=
    \sum_{i=1}^{n}4i^2=
    4\sum_{i=1}^{n}i^2=
    4\frac{n(n+1)(2n+1)}{6}=
    \frac{2n(n+1)(2n+1)}{3}.
  \end{align*}
\end{rem}


\begin{prop}
  \label{p:20230709}
  $\forall n\in\ZZ_{>0}$, $\sum_{i=1}^{2n}(-1)^ii^2=n(2n+1)$.
\end{prop}
\begin{proof**}
  $-1+2^2=3=1\cdot (2\cdot 1+1)$である.
  また,
    $(2n)^2-(2n-1)^2+\sum_{i=1}^{2(n-1)}(-1)^ii^2=\sum_{i=1}^{2n}(-1)^ii$,
    $(2n)^2-(2n-1)^2+(n-1)(2n-1)=n(2n+1)$
  であるので,
  $n$に関する数学的帰納法により示せる.
\end{proof**}
\begin{proof*}
  $P(n)$を次の命題とする:
  \begin{align*}
    \sum_{i=1}^{2n}(-1)^ii^2=n(2n+1)
  \end{align*}
  このとき,
  全ての$n\in\ZZ_{>0}$で$P(n)$が成り立つことを,
  数学的帰納法で示す.

  \paragraph{Base Case:}
  $P(1)$が成り立つことは, 以下から明らか:
  \begin{align*}
    \sum_{i=1}^{2}(-1)^ii^2&=-1+4=3,\\
    1(2\cdot 1+1)&=3.
  \end{align*}

  \paragraph{Induction Step:}
  $P(n-1)\implies P(n)$を示す.
  仮定から$\sum_{i=1}^{2(n-1)}(-1)^ii^2=(n-1)(2n-1)$であるので,
  \begin{align*}
    \sum_{i=1}^{2n}(-1)^ii^2&=(2n)^2-(2n-1)^2+\sum_{i=1}^{2n-2}(-1)^ii^2\\
    &=(2n)^2-(2n-1)^2+(n-1)(2n-1)\\
    &=(2n+2n-1)(2n-2n+1)+(n-1)(2n-1)\\
    &=4n-1+2n^2-3n+1\\
    &=2n^2+n\\
    =n(2n+1)
    .\qedhere
  \end{align*}
\end{proof*}

\begin{rem}
  \cref{p:20230630}をみとめ,
  数学的帰納法を用いず, 以下のように示す方が一般的だと思う:
  \begin{align*}
    \sum_{i=1}^{2n}(-1)^ii^2
    &=\sum_{i=1}^{n}(-1)^{2i-1}(2i-1)^2+\sum_{i=1}^{n}(-1)^{2i}(2i)^2\\
    &=\sum_{i=1}^{n}-(2i-1)^2+\sum_{i=1}^{n}(2i)^2\\
    &=\sum_{i=1}^{n}(-(2i-1)^2+(2i)^2)\\
    &=\sum_{i=1}^{n}(2i-2i+1)(2i+2i-1)\\
    &=\sum_{i=1}^{n}(4i-1)\\
    &=4\sum_{i=1}^{n}i-\sum_{i=1}^{n}1\\
    &=4\frac{n(n+1)}{2}-n\\
    &=2n(n+1)-n\\
    &=n(2(n+1)-1)\\
    &=n(2n+1).
  \end{align*}
\end{rem}


\begin{prop}
  \label{p:20230710}
  $\forall n\in \NN$, $\sum_{i=0}^n i^3=\frac{n^2(n+1)^2}{4}$.
\end{prop}

\begin{proof**}
  $0^3=0=\frac{0^2\cdot 1^2}{4}$である.
  また,
    $n^3+\sum_{i=0}^{n-1}i^3=\sum_{i=0}^{n}i^3$,
    $n^3+\frac{(n-1)^2n^2}{4}=\frac{4n^3+(n-1)^2n^2}{4}=\frac{n^2(4n+(n-1)^2)}{4}=\frac{n^2(n+1)^2}{4}$
  であるので,
  $n$に関する数学的帰納法により示せる.
\end{proof**}

\begin{proof*}
  $P(n)$を次の命題とする:
  \begin{align*}
    \sum_{i=0}^{n}i^3=\frac{n^2(n+1)^2}{4}.
  \end{align*}
  このとき,
  全ての$n\in\NN$で$P(n)$が成り立つことを,
  数学的帰納法で示す.

  \paragraph{Base Case:}
  $P(0)$が成り立つことは, 以下から明らか:
  \begin{align*}
    \sum_{i=0}^{0}i^3&=0,\\
    \frac{0^2(0+1)^2}{4}&=0.
  \end{align*}

  \paragraph{Induction Step:}
  $P(n-1)\implies P(n)$を示す.
  仮定から$\sum_{i=0}^{n-1}i^3=\frac{(n-1)^2n^2}{4}$であるので,
  \begin{align*}
    \sum_{i=0}^{n}i^3
    &=n^3+\sum_{i=0}^{n-1}i^3\\
    &n^3+\frac{(n-1)^2n^2}{4}\\
    &=\frac{4n^3+(n-1)^2n^2}{4}\\
    &=\frac{n^2(4n+(n-1)^2)}{4}\\
    &=\frac{n^2(4n+n^2-2n+1)}{4}\\
    &=\frac{n^2(n^2+2n+1)}{4}\\
    &=\frac{n^2(n+1)^2}{4}
    .\qedhere
  \end{align*}
\end{proof*}

\begin{rem}
  \cref{p:20230630,p:20230705}をみとめ,
  数学的帰納法を用いず, 以下のように示す方が一般的だと思う:
  \begin{align*}
    S&=\sum_{i=0}^n i^3\\
    T&=\sum_{i=1}^{n}(i^4-(i-1)^4)
  \end{align*}
  とする.
  このとき,
  \begin{align*}
    T&=\sum_{i=1}^{n}i^4-\sum_{i=1}^n (i-1)^4\\
    &=\sum_{i=1}^{n}i^4-\sum_{i=0}^{n-1} i^4\\
    &=(n^4+\sum_{i=1}^{n-1}i^4)-(\sum_{i=1}^{n-1} i^4)+0^4)\\
    &=n^4.
  \end{align*}
  一方次のようにも計算できる:
  \begin{align*}
    T
    &=\sum_{i=1}^{n}(i^4- (i^4-4i^3+6i^2-4i+1))\\
    &=\sum_{i=1}^{n}(4i^3-6i^2+4i-1)\\
    &=4\sum_{i=1}^{n}i^3-6\sum_{i=1}^{n}i^2+4\sum_{i=1}^{n}i-\sum_{i=1}^{n}1)\\
    &=4S-6\frac{n(n+1)(2n+1)}{6}+4\frac{n(n+1)}{2}-n\\
    &=4S-n(n+1)(2n+1)+2n(n+1)-n\\
    &=4S+n(-(n+1)(2n+1)+2(n+1)-1)\\
    &=4S+n((n+1)(-(2n+1)+2)-1)\\ 
    &=4S+n((n+1)(-2n+1)-1)\\ 
    &=4S+n(-2n^2-n+1-1)\\ 
    &=4S+n(-2n^2-n)\\ 
    &=4S+n^2(-2n-1)\\ 
 \end{align*}
  したがって,
  \begin{align*}
    n^4&=4S+n^2(-2n-1)\\ 
    4S&=n^4+n^2(2n+1)\\ 
    4S&=n^2(n^2+(2n+1))\\ 
    4S&=n^2(n^2+2n+1)\\ 
    4S&=n^2(n+1)^2\\ 
    S&=\frac{n^2(n+1)^2}{4}.
  \end{align*}
\end{rem}

\begin{prop}
  \label{p:20230715}
  $\forall n\in \NN$,
  $\left(\sum_{i=0}^n i\right)^2=\sum_{i=0}^n i^3$.
\end{prop}

\begin{proof**}
  $0^2=0=0^3$である.
  また,
  \begin{align*}
    \left(\sum_{i=0}^n i\right)^2-\left(\sum_{i=0}^{n-1} i\right)^2
    &=\left(\sum_{i=0}^n i-\sum_{i=0}^{n-1} i\right)\left(\sum_{i=0}^n i+\sum_{i=0}^{n-1} i\right)\\
    &=n\left(-n+2\sum_{i=0}^n i\right)\\
    &=n(-n+n(n+1))\\
    &=n^3,\\
    \sum_{i=0}^n i^3-\sum_{i=0}^{n-1} i^3
    &=n^3
    \end{align*}
  であるので,
  $n$に関する数学的帰納法により示せる.
\end{proof**}
\begin{proof*}
  $P(n)$を次の命題とする:
  \begin{align*}
    \left(\sum_{i=0}^n i\right)^2=\sum_{i=0}^n i^3.
  \end{align*}
  このとき,
  全ての$n\in\NN$で$P(n)$が成り立つことを,
  数学的帰納法で示す.

  \paragraph{Base Case:}
  $P(0)$が成り立つことは, 以下から明らか:
  \begin{align*}
    \left(\sum_{i=0}^0 i\right)^2&=0^2=0\\
    \sum_{i=0}^{0}i^3&=0^3=0.
  \end{align*}

  \paragraph{Induction Step:}
  $P(n-1)\implies P(n)$を示す.
  \begin{align*}
    \left(\sum_{i=0}^n i\right)^2-\left(\sum_{i=0}^{n-1} i\right)^2
    &=\left(\sum_{i=0}^n i-\sum_{i=0}^{n-1} i\right)\left(\sum_{i=0}^n i+\sum_{i=0}^{n-1} i\right)\\
    &=n\left(-n+2\sum_{i=0}^n i\right)\\
    &=n(-n+2\frac{n(n+1)}{2})\\
    &=n(-n+n(n+1))\\
    &=nn^2\\
    &=n^3
  \end{align*}
  である.
  仮定から$\left(\sum_{i=0}^{n-1} i\right)^2=\sum_{i=0}^{n-1} i^3$であるので,
  \begin{align*}
    \left(\sum_{i=0}^{n} i\right)^2\\
    &=n^3+\left(\sum_{i=0}^{n-1} i\right)^2\\
    &=n^3+\sum_{i=0}^{n-1} i^3\\
    &=\sum_{i=0}^{n} i^3
    .\qedhere
  \end{align*}
\end{proof*}


\begin{rem}
  \cref{p:20230630,p:20230710}をみとめ,
  数学的帰納法を用いず, 以下のように示す方が一般的だと思う:
  \begin{align*}
    \sum_{i=0}^{n} i&=\frac{n(n+1)}{2}\\
    \sum_{i=0}^{n}i^3&=\frac{n^2(n+1)^2}{4}
  \end{align*}
  であるので,
  \begin{align*}
    \left(\sum_{i=0}^{n} i\right)^2&=\left(\frac{n(n+1)}{2}\right)^2\\
    &=\sum_{i=0}^{n}i^3.
  \end{align*}
\end{rem}


\begin{prop}
  \label{p:20230716}
  $\forall n\in\NN$, $\sum_{i=0}^n i(i+1)=\frac{n(n+1)(n+2)}{3}$
\end{prop}
\begin{proof**}
  $0(0+1)=0=\frac{0(0+1)(0+2)}{3}$である.
  また,
  $\sum_{i=0}^n i(i+1)-\sum_{i=0}^{n-1} i(i+1)=n(n+1)$,
  $\frac{n(n+1)(n+2)}{3}-\frac{(n-1)n(n+1)}{3}=\frac{n(n+1)((n+2)-(n-1))}{3}=n(n+1)$
  であるので,
  $n$に関する数学的帰納法により示せる.
\end{proof**}

\begin{proof*}
  $P(n)$を次の命題とする:
  \begin{align*}
    \sum_{i=0}^n i(i+1)=\frac{n(n+1)(n+2)}{3}.
  \end{align*}
  このとき,
  全ての$n\in\NN$で$P(n)$が成り立つことを,
  数学的帰納法で示す.

  \paragraph{Base Case:}
  $P(0)$が成り立つことは, 以下から明らか:
    \begin{align*}
      0\cdot 1&=0,\\
      \frac{0\cdot 1\cdot 2}{3}&=0.
  \end{align*}

  \paragraph{Induction Step:}
  $P(n-1)\implies P(n)$を示す.
  仮定から$\sum_{i=0}^{n-1} i(i+1)=\frac{(n-1)n(n+1)}{3}$であるので,
  \begin{align*}
    \sum_{i=0}^n i(i+1)
    &=n(n+1)+\sum_{i=0}^{n-1} i(i+1)\\
    &=n(n+1)+\frac{(n-1)n(n+1)}{3}\\
    &=\frac{3n(n+1)+(n-1)n(n+1)}{3}\\
    &=\frac{n(n+1)(3+(n-1))}{3}\\ 
    &=\frac{n(n+1)(n+2)}{3}
    .\qedhere
  \end{align*}
\end{proof*}
\begin{rem}
  \cref{p:20230630,p:20230705}をみとめ,
  数学的帰納法を用いず, 以下のように示す方が一般的だと思う:
  \begin{align*}
    \sum_{i=0}^n i(i+1)
    &=\sum_{i=0}^n (i^2+i)\\
    &=\sum_{i=0}^n i^2+\sum_{i=0}^n i\\
    &=\frac{n(n+1)(2n+1)}{6}+\frac{n(n+1)}{2}\\
    &=\frac{n(n+1)(2n+1)+3n(n+1)}{6}\\
    &=\frac{n(n+1)((2n+1)+3)}{6}\\
    &=\frac{n(n+1)(2n+4)}{6}\\
    &=\frac{n(n+1)(n+2)}{3}.
  \end{align*}
\end{rem}
\begin{rem}
\cref{p:20230718}
は,
この一般化である.
\end{rem}


\begin{prop}
  \label{p:20230719}
  $\forall n\in \ZZ_{>0}$, $\sum_{i=1}^n (2i-1)2i=\frac{n(n+1)(4n-1)}{3}$.
\end{prop}
\begin{proof**}
  $1\cdot 2=2=\frac{1(1+1)(4-1)}{3}$である.
  また,
  $\sum_{i=1}^n (2i-1)2i-\sum_{i=1}^{n-1} (2i-1)2i=(2n-1)2n$,
  $\frac{n(n+1)(4n-1)}{3}-\frac{(n-1)n(4n-5)}{3}=\frac{n(n+1)(4n-1)-(n-1)n(4n-5)}{3}=
  \frac{n((n+1)(4n-1)-(n-1)(4n-5))}{3}=
  \frac{n(4n^2+3n-1-4n^2+9n-5}{3}=
  \frac{n(12n-6)}{3}=
  \frac{6n(2n-1)}{3}
  2n(2n-1)$
  であるので,
  $n$に関する数学的帰納法により示せる.
\end{proof**}
\begin{proof*}
  $P(n)$を次の命題とする:
  \begin{align*}
    \sum_{i=1}^n (2i-1)2i=\frac{n(n+1)(4n-1)}{3}.
  \end{align*}
  このとき,
  全ての$n\in\ZZ_{>0}$で$P(n)$が成り立つことを,
  数学的帰納法で示す.

  \paragraph{Base Case:}
  $P(1)$が成り立つことは, 以下から明らか:
    \begin{align*}
      1\cdot 2&=2,\\
      \frac{1\cdot 2\cdot 3}{3}&=2.
  \end{align*}

  \paragraph{Induction Step:}
  $P(n-1)\implies P(n)$を示す.
  仮定から$\sum_{i=1}^{n-1} (2i-1)2i=\frac{(n-1)n(4n-5)}{3}$であるので,
  \begin{align*}
    \sum_{i=1}^n (2i-1)2i
    &=(2n-1)2n+\sum_{i=1}^{n-1} (2i-1)2i\\
    &=(2n-1)2n+\frac{(n-1)n(4n-5)}{3}\\
    &=\frac{(2n-1)6n+(n-1)n(4n-5)}{3}\\
    &=\frac{n((2n-1)6+(n-1)(4n-5))}{3}\\
    &=\frac{n(12n-6+4n^2-9n+5}{3}\\
    &=\frac{n(4n^2+3n-1}{3}\\
    &=\frac{n(n+1)(4n+1)}{3}
    .\qedhere
  \end{align*}
\end{proof*}
\begin{rem}
  \cref{p:20230630,p:20230705}をみとめ,
  数学的帰納法を用いず, 以下のように示す方が一般的だと思う:
  \begin{align*}
    \sum_{i=1}^n (2i-1)2i
    &=\sum_{i=1}^n (4i^2-2i)\\
    &=4\sum_{i=1}^n i^2-2\sum_{i=1}^ni)\\
    &=4\frac{n(n+1)(2n+1)}{6}-2\frac{n(n+1)}{2}\\
    &=\frac{2n(n+1)(2n+1)}{3}-n(n+1)\\
    &=\frac{2n(n+1)(2n+1)-3n(n+1)}{3}\\
    &=\frac{n(n+1)(2(2n+1)-3)}{3}\\
    &=\frac{n(n+1)(4n+1)}{3}.
  \end{align*}
\end{rem}


\begin{prop}
  \label{p:20230717}
  $\forall k\in\NN$,
  $\forall n\in\NN$, $\sum_{i=0}^n \binom{i+k}{k}=\binom{n+k+1}{k+1}$
\end{prop}

\begin{proof**}
  $k\in\NN$とする.
  このとき,
  $\binom{k}{k}=1=\binom{k+1}{k+1}$である.
  また,
  $\sum_{i=0}^n \binom{i+k}{k}-\sum_{i=0}^{n-1} \binom{i+k}{k}=\binom{n+k}{k}$,
  $\binom{n+k+1}{k+1}-\binom{n-1+k+1}{k+1}=\binom{n+k+1}{k+1}-\binom{n+k}{k+1}=\binom{n+k}{k}+\binom{n+k}{k+1}-\binom{n+k}{k+1}=\binom{n+k}{k}$
  であるので,
  $n$に関する数学的帰納法により示せる.
\end{proof**}

\begin{proof*}
  $k\in \NN$とする.
  
  $P(n)$を次の命題とする:
  \begin{align*}
    \sum_{i=0}^n \binom{i+k}{k}=\binom{n+k+1}{k+1}.
  \end{align*}
  このとき,
  全ての$n\in\NN$で$P(n)$が成り立つことを,
  数学的帰納法で示す.

  \paragraph{Base Case:}
  $P(0)$が成り立つことは, 以下から明らか:
  \begin{align*}
    \sum_{i=0}^0 \binom{i+k}{k}&=\binom{k}{k}=1\\
    \binom{0+k+1}{k+1}&=\binom{k+1}{k+1}=1.
  \end{align*}

  \paragraph{Induction Step:}
  $P(n-1)\implies P(n)$を示す.
  仮定から$\sum_{i=0}^{n-1} \binom{i+k}{k}=\binom{n+k}{k+1}$であるので,
  \begin{align*}
    \sum_{i=0}^n \binom{i+k}{k}
    &=\binom{n+k}{k} +\sum_{i=0}^{n-1} \binom{i+k}{k}\\
    &=\binom{n+k}{k} + \binom{n+k}{k+1}\\
    &=\binom{n+k}{k} + \binom{n+k}{k+1}\\
    &=\binom{n+k+1}{k+1}
    .\qedhere
  \end{align*}
\end{proof*}

\begin{rem}
  数学的帰納法を用いず, 以下のように示すこともできる:
  \begin{align*}
    X&=\Set{(t_1\ldots,t_k,t_{k+1})|1\leq t_1<\cdots < t_k<t_{k+1} \leq  n+k+1}\\
    X_i&=\Set{(t_1\ldots,t_{k},k+1+i)|1\leq t_1<\cdots < t_k <k+1+i }
  \end{align*}
  とおく. このとき,
  \begin{align*}
    \coprod_{i=0}^{n}X_i = X
  \end{align*}
  であり,
  \begin{align*}
    \numof{X}&=\binom{n+k+1}{k+1}\\
    \numof{X_i}&=\binom{i+k}{k}
  \end{align*}
  であるので,
  \begin{align*}
    \sum_{i=0}^{n}\binom{i+k}{k}=\sum_{i=0}^{n}\numof{X_i}
    =\numof{\coprod_{i=0}^{n}X_i}= \numof{X}=\binom{n+k+1}{k+1}.
  \end{align*}
\end{rem}

\begin{prop}
  \label{p:20230718}
  $\forall m\in \NN$,
  $\forall n\in \NN$,
  $\sum_{i=0}^n \prod_{k=1}^{m}(i+k)=\frac{\prod_{k=1}^{m+1}(n+k)}{m+1}$.
\end{prop}

\begin{proof**}
  $m\in\NN$とする.
  このとき,
  $\prod_{k=1}^{m}(k)=\frac{\prod_{k=1}^{m+1}(k)}{m+1}$である.
  また,
  \begin{align*}
    \sum_{i=0}^n \prod_{k=1}^{m}(i+k)-\sum_{i=0}^{n-1} \prod_{k=1}^{m}(i+k)
    &=\prod_{k=1}^{m}(n+k)\\
  \frac{\prod_{k=1}^{m+1}(n+k)}{m+1}
  -
  \frac{\prod_{k=1}^{m+1}(n-1+k)}{m+1}
  &=
  \frac{\prod_{k=1}^{m+1}(n+k)-\prod_{k=1}^{m+1}(n-1+k)}{m+1}
  \\&=
  \frac{\prod_{k=1}^{m+1}(n+k)-\prod_{k=0}^{m}(n+k)}{m+1} 
  \\&=
  \frac{\prod_{k=1}^{m}(n+k)\cdot((n+m+1)-n)}{m+1}
  \\&=\frac{\prod_{k=1}^{m}(n+k)\cdot(m+1)}{m+1}
  \\&=\prod_{k=1}^{m}(n+k)
  \end{align*}
  であるので,
  $n$に関する数学的帰納法により示せる.
\end{proof**}

\begin{proof*}
  $m\in \NN$とする.
  
  $P(n)$を次の命題とする:
  \begin{align*}
    \sum_{i=0}^n \prod_{k=1}^{m}(i+k)=\frac{\prod_{k=1}^{m+1}(n+k)}{m+1}
  \end{align*}
  このとき,
  全ての$n\in\NN$で$P(n)$が成り立つことを,
  数学的帰納法で示す.

  \paragraph{Base Case:}
  $P(0)$が成り立つことは, 以下から明らか:
  \begin{align*}
    \sum_{i=0}^0 \prod_{k=1}^{m}(i+k)&=\prod_{k=1}^{m}k=m!.\\
    \frac{\prod_{k=1}^{m+1}k}{m+1}&=\frac{(m+1)!}{m+1}=m!.
  \end{align*}

  \paragraph{Induction Step:}
  $P(n-1)\implies P(n)$を示す.
  仮定から$\sum_{i=0}^{n-1} \prod_{k=1}^{m}(i+k)=\frac{\prod_{k=1}^{m+1}(n-1+k)}{m+1}$であるので,
  \begin{align*}
    \sum_{i=0}^n \prod_{k=1}^{m}(i+k)
    &=\prod_{k=1}^{m}(n+k)+\sum_{i=0}^{n-1} \prod_{k=1}^{m}(i+k)\\
    &=\prod_{k=1}^{m}(n+k)+\frac{\prod_{k=1}^{m+1}(n-1+k)}{m+1}\\
    &=\frac{(m+1)\prod_{k=1}^{m}(n+k)+\prod_{k=1}^{m+1}(n-1+k)}{m+1}\\
    &=\frac{(m+1)\prod_{k=1}^{m}(n+k)+\prod_{k=0}^{m}(n+k)}{m+1}\\
    &=\frac{(m+1)+(n))\prod_{k=1}^{m}(n+k)}{m+1}\\
    &=\frac{(n+m+1)\prod_{k=1}^{m}(n+k)}{m+1}\\
    &=\frac{\prod_{k=1}^{m+1}(n+k)}{m+1}
    .\qedhere
  \end{align*}
\end{proof*}

\begin{rem}
    \cref{p:20230717}を認めれば,
  数学的帰納法を用いず, 以下のように示すこともできる:
  \begin{align*}
    \sum_{i=0}^n \binom{i+m}{m}&=\binom{n+m+1}{m+1}\\
    \sum_{i=0}^n \frac{\prod_{k=1}^{m}(i+k)}{m!}&=\frac{\prod_{i=1}^{m+1}(n+k)}{(m+1)!}\\
    m!\sum_{i=0}^n \frac{\prod_{k=1}^{m}(i+k)}{m!}&=m!\frac{\prod_{i=1}^{m+1}(n+k)}{(m+1)!}\\
    \prod_{k=1}^{m}(i+k)&=\frac{\prod_{i=1}^{m+1}(n+k)}{m+1}.
  \end{align*}
\end{rem}



\begin{prop}
  \label{p:20230720}
  $\forall x\neq 0$,
  $\forall n\in\NN$,
  $\sum_{i=0}^{n}x^i=\frac{1-x^{n+1}}{1-x}$.
\end{prop}

\begin{proof**}
  $x\neq 0$とする.
  このとき,
  $x^0=1=\frac{x-1}{x-1}$である.
  また,
  \begin{align*}
    \sum_{i=0}^{n}x^i-\sum_{i=0}^{n-1}x^i&=x^n\\
    \frac{1-x^{n+1}}{1-x}-\frac{1-x^{n}}{1-x}
    &=\frac{-x^{n+1}+x^n}{1-x}
    =\frac{x^n(-x+1)}{1-x}
    =x^n
  \end{align*}
  であるので,
  $n$に関する数学的帰納法により示せる.
\end{proof**}

\begin{proof*}
  $x\neq 0$とする.
  
  $P(n)$を次の命題とする:
  \begin{align*}
    \sum_{i=0}^{n}x^i=\frac{1-x^{n+1}}{1-x}.
  \end{align*}
  このとき,
  全ての$n\in\NN$で$P(n)$が成り立つことを,
  数学的帰納法で示す.

  \paragraph{Base Case:}
  $P(0)$が成り立つことは, 以下から明らか:
  \begin{align*}
    \sum_{i=0}^0 x^i&=1,\\
    \frac{1-x^1}{1-x}&=1.
  \end{align*}

  \paragraph{Induction Step:}
  $P(n-1)\implies P(n)$を示す.
  仮定から$    \sum_{i=0}^{n-1}x^i=\frac{1-x^{n}}{1-x}$であるので,
  \begin{align*}
    \sum_{i=0}^{n}x^i
    &=x^n+\sum_{i=0}^{n-1}x^i\\
    &=x^n+\frac{1-x^{n}}{1-x}\\
    &=\frac{x^n(1-x)+1-x^{n}}{1-x}\\
    &=\frac{x^n-x^{n+1}+1-x^{n}}{1-x}\\
    &=\frac{-x^{n+1}+1}{1-x}\\
    &=\frac{1-x^{n+1}}{1-x}
    .\qedhere
  \end{align*}
\end{proof*}

\begin{rem}
  数学的帰納法を用いず, 以下のように示す方が一般的だと思う:
  $x\neq 1$とし,
  $S=\sum_{i=0}^{n}x^i$とおく.
  \begin{align*}
    (1-x)S
    &=\sum_{i=0}^{n}x^i-x\sum_{i=0}^{n}x^i\\
    &=\sum_{i=0}^{n}x^i-\sum_{i=0}^{n}x^{i+1}\\
    &=\sum_{i=0}^{n}x^i-\sum_{i=1}^{n+1}x^{i}\\
    &=x^0-x^{n+1}+\sum_{i=1}^{n}x^i-\sum_{i=1}^{n}x^{i}\\
    &=1-x^{n+1}.\\
    S&=\frac{1-x^{n+1}}{1-x}.
  \end{align*}
\end{rem}

\begin{prop}
  \label{p:20230721}
  $\forall x\neq 0$,
  $\forall n\in\NN$,
  $\sum_{i=0}^n ix^i=\frac{nx^{n+2}-(n+1)x^{n+1}+x}{(x-1)^2}$.
\end{prop}

\begin{proof**}
  $x\neq 0$とする.
  このとき,
  $x^0=1$.
  $\frac{0x^{2}-x^{1}+x}{(x-1)^2}=0$である.
  また,
  \begin{align*}
    \sum_{i=0}^{n}ix^i-\sum_{i=0}^{n-1}ix^i&=nx^n\\
    \frac{nx^{n+2}-(n+1)x^{n+1}+x}{(x-1)^2}-\frac{(n-1)x^{n+1}-nx^{n}+x}{(x-1)^2}
    &=
    \frac{nx^{n+2}-(n+1)x^{n+1}+x-((n-1)x^{n+1}-nx^{n}+x)}{(x-1)^2}\\
    &=
    \frac{nx^{n+2}-(n+1)x^{n+1}+x-(n-1)x^{n+1}+nx^{n}-x}{(x-1)^2}\\
    &=
    \frac{nx^{n+2}-2nx^{n+1}+nx^{n}}{(x-1)^2}\\
    &=
    \frac{nx^{n}(x^2-2x+1)}{(x-1)^2}\\
    &=nx^n
  \end{align*}
  であるので,
  $n$に関する数学的帰納法により示せる.
\end{proof**}


\begin{proof*}
  $x\neq 0$とする.
  
  $P(n)$を次の命題とする:
  \begin{align*}
    \sum_{i=0}^n ix^i=\frac{nx^{n+2}-(n+1)x^{n+1}+x}{(x-1)^2}.
  \end{align*}
  このとき,
  全ての$n\in\NN$で$P(n)$が成り立つことを,
  数学的帰納法で示す.

  \paragraph{Base Case:}
  $P(0)$が成り立つことは, 以下から明らか:
  \begin{align*}
    \sum_{i=0}^0 ix^i&=0,\\
    \frac{0x^{0+2}-(0+1)x^{0+1}+x}{(x-1)^2}&=\frac{0-x+x}{(x-1)^2}=0.
  \end{align*}

  \paragraph{Induction Step:}
  $P(n-1)\implies P(n)$を示す.
  仮定から$\sum_{i=0}^{n-1} ix^i=\frac{(n-1)x^{n+1}-nx^{n}+x}{(x-1)^2}$であるので,
  \begin{align*}
    \sum_{i=0}^n ix^i
    &=nx^n+\sum_{i=0}^{n-1} ix^i\\
    &=nx^n+\frac{(n-1)x^{n+1}-nx^{n}+x}{(x-1)^2}\\
    &=\frac{nx^n(x-1)^2+(n-1)x^{n+1}-nx^{n}+x}{(x-1)^2}\\
    &=\frac{nx^n(x^2-2x+1)+(n-1)x^{n+1}-nx^{n}+x}{(x-1)^2}\\
    &=\frac{nx^{n+2}-2nx^{n+1}+nx^n+(n-1)x^{n+1}-nx^{n}+x}{(x-1)^2}\\
    &=\frac{nx^{n+2}-(n+1)x^{n+1}+x}{(x-1)^2}
    .\qedhere
  \end{align*}
\end{proof*}
\begin{rem}
  \cref{p:20230720}を認めて,
  数学的帰納法を用いず, 以下のように示す方が一般的だと思う:
  $x\neq 1$とし,
  $S=\sum_{i=0}^{n}ix^i$とおく.
  \begin{align*}
    (1-x)S
    &=\sum_{i=0}^{n}ix^i-x\sum_{i=0}^{n}ix^i\\
    &=\sum_{i=0}^{n}ix^i-\sum_{i=0}^{n}ix^{i+1}\\
    &=\sum_{i=0}^{n}ix^i-\sum_{i=1}^{n+1}(i-1)x^{i}\\
    &=0x^0-nx^{n+1}+\sum_{i=1}^{n}ix^i-\sum_{i=1}^{n}(i-1)x^{i}\\
    &=-nx^{n+1}+\sum_{i=1}^{n}(ix^i-(i-1)x^{i})\\
    &=-nx^{n+1}+\sum_{i=1}^{n}x^{i}\\
    &=-nx^{n+1}-1+\sum_{i=0}^{n}x^{i}\\
    &=-nx^{n+1}-1+\frac{x^{n+1}-1}{x-1}\\
    &=\frac{-(nx^{n+1}+1)(x-1)+x^{n+1}-1}{x-1}\\
    &=\frac{-nx^{n+2}-x+nx^{n+1}+1+x^{n+1}-1}{x-1}\\
    &=\frac{-nx^{n+2}+(n+1)x^{n+1}-x}{x-1}.\\
    S&=\frac{-nx^{n+2}+(n+1)x^{n+1}-x}{(1-x)(x-1)}\\
    &=\frac{nx^{n+2}-(n+1)x^{n+1}+x)}{(1-x)^2}.
  \end{align*}
\end{rem}
\begin{rem}
  \cref{p:20230720}を認めて,
  数学的帰納法を用いず, 以下のように示す方が一般的だと思う:
  $x\neq 1$とし,
  $S(x)=\sum_{i=0}^{n}x^i$とおく.
  \begin{align*}
    S(x)&=\sum_{i=0}^{n}x^i.&
    S(x)&=\frac{x^{n+1}-1}{x-1}.\\
    \frac{d}{dx}S(x)&=\sum_{i=0}^{n}ix^{i-1}.&
    \frac{d}{dx}S(x)
    &=\frac{(n+1)x^{n}(x-1)-(x^{n+1}-1)}{(x-1)^2}\\
    &&&=\frac{(n+1)(x^{n+1}-x^{n})-x^{n+1}+1}{(x-1)^2}\\
    &&&=\frac{nx^{n+1}-(n+1)x^{n}+1}{(x-1)^2}.\\
    x\frac{d}{dx}S(x)&=\sum_{i=0}^{n}ix^i.&
    x\frac{d}{dx}S(x)
    &=\frac{nx^{n+2}-(n+1)x^{n+1}+x}{(x-1)^2}.
  \end{align*}
\end{rem}
    
\begin{prop}
  \label{p:20230722}
  $\forall n\in \ZZ_{>0}$,
  $\sum_{i=1}^n (3i-2)2^{i-1}=(3n-5)2^n+5$.
\end{prop}

\begin{proof**}
  $(3-2)2^{1-1}=1=(3-5)2^1+5$である.
  また,
  \begin{align*}
    \sum_{i=1}^n (3i-2)2^{i-1}-\sum_{i=1}^{n-1} (3i-2)2^{i-1}&=(3n-2)2^{n-1}\\
    (3n-5)2^n+5-((3(n-1)-5)2^{n-1}+5)
    &=(2(3n-5)-(3(n-1)-5))2^{n-1}\\
    &=(6n-10-3n+8)2^{n-1}\\
    &=(3n-2)2^{n-1}
  \end{align*}
  であるので,
  $n$に関する数学的帰納法により示せる.
\end{proof**}

\begin{proof*}
  $P(n)$を次の命題とする:
  \begin{align*}
    \sum_{i=1}^n (3i-2)2^{i-1}=(3n-5)2^n+5.
  \end{align*}
  このとき,
  全ての$n\in\ZZ_{>0}$で$P(n)$が成り立つことを,
  数学的帰納法で示す.

  \paragraph{Base Case:}
  $P(1)$が成り立つことは, 以下から明らか:
  \begin{align*}
    \sum_{i=0}^1 (3i-2)2^{i-1}&=(3i-1)2^{1-1}=1,\\
    (3n-5)2^n+5&=(3-5)2^1+5=-4+5=1.
  \end{align*}

  \paragraph{Induction Step:}
  $P(n-1)\implies P(n)$を示す.
  仮定から$\sum_{i=1}^{n-1} (3i-2)2^{i-1}=(3n-8)2^{n-1}+5$であるので,
  \begin{align*}
    \sum_{i=1}^n (3i-2)2^{i-1}
    &=(3n-2)2^{n-1}+\sum_{i=1}^{n-1} (3i-2)2^{i-1}\\
    &=(3n-2)2^{n-1}+(3n-8)2^{n-1}+5\\
    &=2^{n-1}3n-2^{n-1}2+2^{n-1}3n-2^{n-1}8+5\\
    &=2^{n-1}3n+2^{n-1}3n-2^{n-1}2-2^{n-1}8+5\\
    &=2\cdot 2^{n-1}3n-2^{n-1}(2+8)+5\\
    &=2\cdot 2^{n-1}3n-2^{n-1}10+5\\
    &=2^{n}3n-2^{n}5+5\\
    &=(3n-5)2^n+5
    .\qedhere
  \end{align*}
\end{proof*}
\begin{rem}
  \cref{p:20230720}を認めて,
  数学的帰納法を用いず, 以下のように示す方が一般的だと思う:
  $S=\sum_{i=1}^n (3i-2)2^{i-1}$とおく.
  \begin{align*}
    (1-2)S
    &=\sum_{i=1}^n (3i-2)2^{i-1}-2\sum_{i=1}^n (3i-2)2^{i-1}\\
    &=\sum_{i=1}^n (3i-2)2^{i-1}-\sum_{i=1}^n (3i-2)2^{i}\\
    &=\sum_{i=0}^{n-1} (3i+1)2^{i}-\sum_{i=1}^n (3i-2)2^{i}\\
    &=(3\cdot0+1)2^{0}-(3n-2)2^{n}+\sum_{i=1}^{n-1} (3i+1)2^{i}-\sum_{i=1}^{n-1} (3i-2)2^{i}\\
    &=1-(3n-2)2^{n}+\sum_{i=1}^{n-1} ((3i+1)2^{i}-(3i-2)2^{i})\\
    &=1-(3n-2)2^{n}+\sum_{i=1}^{n-1}3\cdot 2^{i}\\
    &=1-(3n-2)2^{n}+3\cdot 2\sum_{i=1}^{n-1}\cdot 2^{i-1}\\
    &=1-(3n-2)2^{n}+3\cdot 2\sum_{i=0}^{n-2}\cdot 2^{i}\\
    &=1-(3n-2)2^{n}+3\cdot 2\frac{1-2^{n-1}}{1-2}\\
    &=1-(3n-2)2^{n}-6+3\cdot 2^{n}\\
    &=-5+(-3n+5)2^{n}.\\
    S&=5+(3n-5)2^{n}.
  \end{align*}
\end{rem}


\subsubsection{整数や整式の積に関するもの}
\begin{prop}
  \label{p:20230723}
  $\forall n\in \NN$,
  $(2n)!!=2^nn!$.
\end{prop}

\begin{proof**}
  $0!!=1=2^00!$である.
  また,
  \begin{align*}
    \frac{(2n)!!}{(2n-2)!!}&=2n\\
    \frac{2^nn!}{2^{n-1}(n-1)!}&=\frac{2^n}{2^{n-1}}\frac{n!}{(n-1)!}=2n
  \end{align*}
  であるので,
  $n$に関する数学的帰納法により示せる.
\end{proof**}


\begin{proof*}
  $P(n)$を次の命題とする:
  \begin{align*}
    (2n)!!=2^nn!.
  \end{align*}
  このとき,
  全ての$n\in\NN$で$P(n)$が成り立つことを,
  数学的帰納法で示す.

  \paragraph{Base Case:}
  $P(0)$が成り立つことは, 以下から明らか:
  \begin{align*}
    0!!&=1.\\
    2^00!&=1.
  \end{align*}

  \paragraph{Induction Step:}
  $P(n-1)\implies P(n)$を示す.
  仮定から$(2n-2)!!=2^{n-1}(n-1)!$であるので,
  \begin{align*}
    (2n)!!
    &=2n(2n-2)!!\\
    &=2n2^{n-1}(n-1)!\\
    &=2^{n}n(n-1)!\\
    &=2^nn!
    .\qedhere
  \end{align*}
\end{proof*}

\begin{rem}
  数学的帰納法を用いず, 以下のように示す方が一般的だと思う:
  $n=0$のときは, 定義から$0!!=1=2^00!$.
  $n>0$のときは,
  \begin{align*}
    (2n)!!
    &=\prod_{i=1}^{n}(2i)\\
    &=\prod_{i=1}^{n}2\prod_{i=1}^{n}i\\
    &=2^nn!.
  \end{align*}
\end{rem}


\begin{prop}
  \label{p:20230724}
  $\forall n\in\NN$,
  $(2n+1)!!=\frac{(2n+1)!}{2^nn!}$.
\end{prop}
\begin{proof**}
  $1!!=1=\frac{1}{2^0\cdot 1!}$である.
  また,
  \begin{align*}
    \frac{(2n+1)!!}{(2n-1)!!}&=2n+1\\
    \frac{\frac{(2n+1)!}{2^nn!}}{\frac{(2n-1)!}{2^{n-1}(n-1)!}}
    &=
    \frac{(2n+1)!\cdot 2^{n-1}(n-1)!}{2^nn! \cdot (2n-1)!}
    =\frac{(2n+1)!}{(2n-1)!}\frac{(n-1)!}{n!}\frac{2^{n-1}}{2^n}
    =\frac{2n(2n+1)\cdot 1\cdot 1}{1 \cdot 2 \cdot n}=2n+1
  \end{align*}
  であるので,
  $n$に関する数学的帰納法により示せる.
\end{proof**}

\begin{proof*}
  $P(n)$を次の命題とする:
  \begin{align*}
    (2n+1)!!=\frac{(2n+1)!}{2^nn!}.
  \end{align*}
  このとき,
  全ての$n\in\NN$で$P(n)$が成り立つことを,
  数学的帰納法で示す.

  \paragraph{Base Case:}
  $P(0)$が成り立つことは, 以下から明らか:
  \begin{align*}
    1!!&=1.\\
    \frac{(0+1)!}{2^00!}&=1.
  \end{align*}

  \paragraph{Induction Step:}
  $P(n-1)\implies P(n)$を示す.
  仮定から$(2n-1)!!=\frac{(2n-1)!}{2^{n-1}(n-1)!}$であるので,
  \begin{align*}
    (2n+1)!!
    &=(2n+1)\cdot (2n-1)!!\\
    &=(2n+1)\cdot \frac{(2n-1)!}{2^{n-1}(n-1)!}\\
    &=\frac{(2n+1)2n}{2n}\cdot \frac{(2n-1)!}{2^{n-1}(n-1)!}\\
    &=\frac{(2n+1)!}{2^{n}n!}
    .\qedhere
  \end{align*}
\end{proof*}

\begin{rem}
  \Cref{p:20230723}をみとめて,
  数学的帰納法を用いず, 以下のように示す方が一般的だと思う:
  \begin{align*}
    (2n+1)!!=\frac{(2n+1)!}{(2n)!!}=\frac{(2n+1)!}{2^nn!}.
  \end{align*}
\end{rem}

\begin{prop}
  \label{p:20230725}
  $\forall n\in\ZZ_{>0}$,
  $\frac{(2n)!}{n!}=2^n(2n-1)!!$.
\end{prop}
\begin{proof**}
  $\frac{2!}{1!}=2=2^11!$である.
  また,
  \begin{align*}
    \frac{\frac{(2n)!}{n!}}{\frac{(2n-2)!}{(n-1)!}}
    &=\frac{(2n)!(n-1)!}{(2n-2)!n!}
    =\frac{2n\cdot 2n-1}{n}=2(2n-1)\\
    \frac{2^n(2n-1)!!}{2^{n-1}(2n-3)!!}
    &=2(2n-1)
  \end{align*}
  であるので,
  $n$に関する数学的帰納法により示せる.
\end{proof**}
\begin{proof*}
  $P(n)$を次の命題とする:
  \begin{align*}
    \frac{(2n)!}{n!}=2^n(2n-1)!!.
  \end{align*}
  このとき,
  全ての$n\in\ZZ_{>0}$で$P(n)$が成り立つことを,
  数学的帰納法で示す.

  \paragraph{Base Case:}
  $P(1)$が成り立つことは, 以下から明らか:
  \begin{align*}
    \frac{2!}{1!}&=2,\\
    2^11!&=2.
  \end{align*}

  \paragraph{Induction Step:}
  $P(n-1)\implies P(n)$を示す.
  仮定から$\frac{(2(n-1))!}{(n-1)!}=2^{n-1}(2n-3)!!$であるので,
  \begin{align*}
    \frac{(2n)!}{n!}
    &=\frac{2n\cdot (2n-1)}{n}\frac{(2n-2)!}{(n-1)!}\\
    &=2(2n-1)\frac{(2(n-1))!}{(n-1)!}\\
    &=2(2n-1)2^{n-1}(2n-3)!!
    &=2^n(2n-1)!!
    .\qedhere
  \end{align*}
\end{proof*}
\begin{rem}
  \Cref{p:20230723}をみとめて,
  数学的帰納法を用いず, 以下のように示す方が一般的だと思う:
  \begin{align*}
    n!\cdot 2^n (2n-1)!!
    &=2^n n!\cdot (2n-1)!!
    =(2n)!! (2n-1)!!
    =(2n)!.\\
    2^n (2n-1)!!&=\frac{(2n)!}{n!}.
  \end{align*}
\end{rem}


\subsubsection{有理式などの和に関するもの}
\begin{prop}
  \label{p:20230726}
  $\forall n \in \ZZ_{>0}$,
  $\sum_{i=1}^{2n}\frac{(-1)^{i+1}}{i}=\sum_{i=1}^n\frac{1}{n+i}$.
\end{prop}
\begin{proof**}
  $\frac{1}{1}-\frac{1}{2}=\frac{1}{2}=\frac{1}{1+1}$である.
  また,
  \begin{align*}
    \sum_{i=1}^{2n}\frac{(-1)^{i+1}}{i}-\sum_{i=1}^{2n-2}\frac{(-1)^{i+1}}{i}
    &=\frac{1}{2n-1}-\frac{1}{2n}\\
    &=\frac{2n-(2n-1)}{2n(2n-1)}\\
    &=\frac{1}{2n(2n-1)}\\
    \sum_{i=1}^n\frac{1}{n+i}-\sum_{i=1}^{n-1}\frac{1}{n-1+i}
    &=\sum_{i=1}^n\frac{1}{n+i}-\sum_{i=0}^{n-2}\frac{1}{n+i}\\
    &=(\frac{1}{2n}+\frac{1}{2n-1}+\sum_{i=1}^{n-2}\frac{1}{n+i})
    -(\frac{1}{n}+\sum_{i=0}^{n-2}\frac{1}{n+i})\\
    &=\frac{1}{2n}+\frac{1}{2n-1}-\frac{1}{n}\\
    &=\frac{2n-1+2n-2(2n-1)}{2n(2n-1)}\\
    &=\frac{1}{2n(2n-1)}
  \end{align*}
  であるので,
  $n$に関する数学的帰納法により示せる.
\end{proof**}
\begin{proof*}
  $P(n)$を次の命題とする:
  \begin{align*}
    \sum_{i=1}^{2n}\frac{(-1)^{i+1}}{i}=\sum_{i=1}^n\frac{1}{n+i}.
  \end{align*}
  このとき,
  全ての$n\in\ZZ_{>0}$で$P(n)$が成り立つことを,
  数学的帰納法で示す.

  \paragraph{Base Case:}
  $P(1)$が成り立つことは, 以下から明らか:
  \begin{align*}
    \frac{1}{1}-\frac{1}{2}&=\frac{1}{2}.\\
    \frac{1}{1+1}&=\frac{1}{2}.
  \end{align*}

  \paragraph{Induction Step:}
  $P(n-1)\implies P(n)$を示す.
  仮定から$\sum_{i=1}^{2n-2}\frac{(-1)^{i+1}}{i}=\sum_{i=1}^{n-1}\frac{1}{n+i-1}$であるので,
  \begin{align*}
    \sum_{i=1}^{2n}\frac{(-1)^{i+1}}{i}
    &=-\frac{1}{2n}+\frac{1}{2n-1}+\sum_{i=1}^{2n-2}\frac{(-1)^{i+1}}{i}\\
    &=-\frac{1}{2n}+\frac{1}{2n-1}+\sum_{i=1}^{n-1}\frac{1}{n+i-1}\\
    &=-\frac{1}{2n}+\frac{1}{2n-1}+\sum_{i=0}^{n-2}\frac{1}{n+i}\\
    &=-\frac{1}{2n}+\sum_{i=0}^{n-1}\frac{1}{n+i}\\
    &=-\frac{1}{2n}+\frac{1}{n}+\sum_{i=1}^{n-1}\frac{1}{n+i}\\
    &=\frac{1}{2n}+\sum_{i=1}^{n-1}\frac{1}{n+i}\\
    &=\sum_{i=1}^{n}\frac{1}{n+i}
    .\qedhere
  \end{align*}
\end{proof*}

\begin{prop}
  \label{p:20230727}
  $\forall n \in \ZZ_{>0}$,
  $\sum_{i=1}^n\frac{1}{i(i+1)}=\frac{n}{n+1}$.
\end{prop}
\begin{proof**}
  $\frac{1}{1(2)}=\frac{1}{2}=\frac{1}{1+1}$である.
  また,
  \begin{align*}
    \sum_{i=1}^n\frac{1}{i(i+1)}-\sum_{i=1}^{n-1}\frac{1}{i(i+1)}
    &=\frac{1}{n(n+1)}\\
    \frac{n}{n+1}-\frac{n-1}{n}
    &=\frac{n^2-(n+1)(n-1)}{n(n+1)}
    =\frac{n^2-(n^2-1)}{n(n+1)}
    =\frac{1}{n(n+1)}
  \end{align*}
  であるので,
  $n$に関する数学的帰納法により示せる.
\end{proof**}

\begin{proof*}
  $P(n)$を次の命題とする:
  \begin{align*}
    \sum_{i=1}^n\frac{1}{i(i+1)}=\frac{n}{n+1}.
  \end{align*}
  このとき,
  全ての$n\in\ZZ_{>0}$で$P(n)$が成り立つことを,
  数学的帰納法で示す.

  \paragraph{Base Case:}
  $P(1)$が成り立つことは, 以下から明らか:
  \begin{align*}
    \frac{1}{1(1+1)}&=\frac{1}{2}.\\
    \frac{1}{1+1}&=\frac{1}{2}.
  \end{align*}

  \paragraph{Induction Step:}
  $P(n-1)\implies P(n)$を示す.
  仮定から$\sum_{i=1}^{n-1}\frac{1}{i(i+1)}=\frac{n-1}{n}$であるので,
  \begin{align*}
    \sum_{i=1}^n\frac{1}{i(i+1)}
    &=\frac{1}{n(n+1)}+\sum_{i=1}^{n-1}\frac{1}{i(i+1)}\\
    &=\frac{1}{n(n+1)}+\frac{n-1}{n}\\
    &=\frac{1+(n+1)(n-1)}{n(n+1)}\\
    &=\frac{1+n^2-1}{n(n+1)}\\
    &=\frac{n^2}{n(n+1)}\\
    &=\frac{n}{(n+1)}
    .\qedhere
  \end{align*}
\end{proof*}
\begin{rem}
  数学的帰納法を用いず, 以下のように示す方が一般的だと思う:
  \begin{align*}
    \sum_{i=1}^n\frac{1}{i(i+1)}
    &=\sum_{i=1}^n(\frac{1}{i}-\frac{1}{i+1})\\
    &=\sum_{i=1}^n\frac{1}{i}-\sum_{i=1}^n\frac{1}{i+1}\\
    &=\sum_{i=1}^n\frac{1}{i}-\sum_{i=2}^{n+1}\frac{1}{i}\\
    &=(1+\sum_{i=2}^n\frac{1}{i})-(\frac{1}{n+1}+\sum_{i=2}^{n}\frac{1}{i})\\
    &=1-\frac{1}{n+1}\\
    &=\frac{n}{n+1}.
  \end{align*}
\end{rem}

\begin{prop}
  \label{p:20230728}
  $\forall m \in \ZZ_{>0}$,
  $\forall n \in \ZZ_{>0}$,
  $\sum_{i=1}^n\prod_{k=0}^{m}\frac{1}{i+k}=\frac{1}{m}\left(\frac{1}{m!}-\prod_{k=1}^m\frac{1}{n+k}\right)$.
\end{prop}
\begin{proof**}
  \begin{align*}
    \prod_{k=0}^{m}\frac{1}{1+k}&=\frac{1}{(m+1)!}\\
    \frac{1}{m}\left(\frac{1}{m!}-\prod_{k=1}^m\frac{1}{1+k}\right)
    &=\frac{1}{m}\left(\frac{1}{m!}-\frac{1}{(1+m)!}\right)
    =\frac{1}{m}\left(\frac{m+1}{(m+1)!}-\frac{1}{(1+m)!}\right)
    =\frac{1}{m}\frac{m}{(1+m)!}=\frac{1}{(m+1)!}
  \end{align*}
  である.
  また,
  \begin{align*}
    \sum_{i=1}^n\prod_{k=0}^{m}\frac{1}{i+k}
    -\sum_{i=1}^{n-1}\prod_{k=0}^{m}\frac{1}{i+k}
    &=\prod_{k=0}^{m}\frac{1}{n+k}\\
    \frac{1}{m}\left(\frac{1}{m!}-\prod_{k=1}^m\frac{1}{n+k}\right)
    -\frac{1}{m}\left(\frac{1}{m!}-\prod_{k=1}^m\frac{1}{n-1+k}\right)
    &=\frac{1}{m}\left(-\prod_{k=1}^m\frac{1}{n+k}+\prod_{k=1}^m\frac{1}{n-1+k}\right)\\
    &=\frac{1}{m}\left(\frac{-n+n+m}{\prod_{k=0}^m n+k}\right)\\
    &=\frac{1}{m}\frac{m}{(n+m)!}\\
    &=\frac{1}{(n+m)!}
  \end{align*}
  であるので,
  $n$に関する数学的帰納法により示せる.
\end{proof**}

\begin{proof*}
  $\forall m \in \ZZ_{>0}$とする.
  $P(n)$を次の命題とする:
  \begin{align*}
   \sum_{i=1}^n\prod_{k=0}^{m}\frac{1}{i+k}=\frac{1}{m}\left(\frac{1}{m!}-\prod_{k=1}^m\frac{1}{n+k}\right).
  \end{align*}
  このとき,
  全ての$n\in\ZZ_{>0}$で$P(n)$が成り立つことを,
  数学的帰納法で示す.

  \paragraph{Base Case:}
  $P(1)$が成り立つことは, 以下から明らか:
  \begin{align*}
    \prod_{k=0}^{m}\frac{1}{1+k}&=\frac{1}{(m+1)!}.\\
    \frac{1}{m}\left(\frac{1}{m!}-\prod_{k=1}^m\frac{1}{1+k}\right)
    &=\frac{1}{m}\left(\frac{1}{m!}-\frac{1}{(1+m)!}\right)
    =\frac{1}{m}\left(\frac{m+1}{(m+1)!}-\frac{1}{(1+m)!}\right)
    =\frac{1}{m}\frac{m}{(1+m)!}=\frac{1}{(m+1)!}.
  \end{align*}
  
  \paragraph{Induction Step:}
  $P(n-1)\implies P(n)$を示す.
  仮定から$\sum_{i=1}^{n-1}\prod_{k=0}^{m}\frac{1}{i+k}=\frac{1}{m}\left(\frac{1}{m!}-\prod_{k=1}^m\frac{1}{n-1+k}\right)$であるので,
  \begin{align*}
    \sum_{i=1}^n\prod_{k=0}^{m}\frac{1}{i+k}
    &=\prod_{k=0}^{m}\frac{1}{n+k}+\sum_{i=1}^{n-1}\prod_{k=0}^{m}\frac{1}{i+k}\\
    &=\prod_{k=0}^{m}\frac{1}{n+k}+\frac{1}{m}\left(\frac{1}{m!}-\prod_{k=1}^m\frac{1}{n-1+k}\right)\\
    &=\frac{1}{\prod_{k=0}^{m} (n+k)}+\frac{1}{m}\left(\frac{1}{m!}-\frac{1}{\prod_{k=1}^m(n-1+k)}\right)\\
    &=\frac{1}{m}\left(\frac{1}{m!}+\frac{m}{\prod_{k=0}^{m} (n+k)}-\frac{1}{\prod_{k=1}^m(n-1+k)}\right)\\
    &=\frac{1}{m}\left(\frac{1}{m!}+\frac{m}{\prod_{k=0}^{m} (n+k)}-\frac{1}{\prod_{k=0}^{m-1}(n+k)}\right)\\
    &=\frac{1}{m}\left(\frac{1}{m!}+\frac{m}{\prod_{k=0}^{m} (n+k)}-\frac{n+m}{\prod_{k=0}^{m}(n+k)}\right)\\
    &=\frac{1}{m}\left(\frac{1}{m!}+\frac{m-(n+m)}{\prod_{k=0}^{m} (n+k)}\right)\\
    &=\frac{1}{m}\left(\frac{1}{m!}-\frac{n}{\prod_{k=0}^{m} (n+k)}\right)\\
    &=\frac{1}{m}\left(\frac{1}{m!}-\frac{1}{\prod_{k=1}^{m} (n+k)}\right)\\
    &=\frac{1}{m}\left(\frac{1}{m!}-\prod_{k=1}^m\frac{1}{n+k}\right)
    .\qedhere
  \end{align*}
\end{proof*}
\begin{proof**}
  \Cref{p:20230727}から,
  \begin{align*}
    \sum_{i=1}^n\frac{1}{i(i+1)}
    &=\frac{n}{n+1}\\
    \frac{1}{1}\left(\frac{1}{1!}-\prod_{k=1}^1\frac{1}{n+k}\right)
    &=
    \left(1-\frac{1}{n+1}\right)
    =\frac{n}{n+1}
  \end{align*}
  である.
  また,
  \begin{align*}
    \sum_{i=1}^n\prod_{k=0}^{m}\frac{1}{i+k}
    &=
    \sum_{i=1}^n\frac{1}{\prod_{k=0}^{m}(i+k)}\\
    &=
    \frac{1}{m}\sum_{i=1}^n\frac{m}{\prod_{k=0}^{m}(i+k)}\\
    &=
    \frac{1}{m}\sum_{i=1}^n\frac{m+i-i}{\prod_{k=0}^{m}(i+k)}\\
    &=
    \frac{1}{m}\sum_{i=1}^n\left(\frac{m+i}{\prod_{k=0}^{m}(i+k)}-\frac{i}{\prod_{k=0}^{m}(i+k)}\right)\\
    &=
    \frac{1}{m}\sum_{i=1}^n\left(\frac{1}{\prod_{k=0}^{m-1}(i+k)}-\frac{1}{\prod_{k=1}^{m}(i+k)}\right)\\
    &=
    \frac{1}{m}\sum_{i=1}^n\frac{1}{\prod_{k=0}^{m-1}(i+k)}-\frac{1}{m}\sum_{i=1}^n\frac{1}{\prod_{k=1}^{m}(i+k)}.\\
    &=
    \frac{1}{m}\sum_{i=1}^n\frac{1}{\prod_{k=0}^{m-1}(i+k)}-\frac{1}{m}\sum_{i=2}^{n+1}\frac{1}{\prod_{k=1}^{m}(i-1+k)}\\
    &=
    \frac{1}{m}\sum_{i=1}^n\frac{1}{\prod_{k=0}^{m-1}(i+k)}-\frac{1}{m}\sum_{i=2}^{n+1}\frac{1}{\prod_{k=0}^{m-1}(i+k)}\\
    &=
    \frac{1}{m}\sum_{i=1}^n\frac{1}{\prod_{k=0}^{m-1}(i+k)}-\frac{1}{m}\sum_{i=1}^{n+1}\frac{1}{\prod_{k=0}^{m-1}(i+k)}
    +\frac{1}{m}\frac{1}{\prod_{k=0}^{m-1}(1+k)}\\
    &=
    \frac{1}{m}\left(\sum_{i=1}^n\frac{1}{\prod_{k=0}^{m-1}(i+k)}-\sum_{i=1}^{n+1}\frac{1}{\prod_{k=0}^{m-1}(i+k)}
    +\frac{1}{\prod_{k=0}^{m-1}(1+k)}\right)\\
    &=
    \frac{1}{m}\left(\sum_{i=1}^n\frac{1}{\prod_{k=0}^{m-1}(i+k)}-\sum_{i=1}^{n+1}\frac{1}{\prod_{k=0}^{m-1}(i+k)}
    +\frac{1}{m!}\right)
  \end{align*}
  であるが,
  \begin{align*}
    &\frac{1}{m}\left(
    \frac{1}{m-1}\left(\frac{1}{(m-1)!}-\prod_{k=1}^{m-1}\frac{1}{n+k}\right)
    -\frac{1}{m-1}\left(\frac{1}{(m-1)!}-\prod_{k=1}^{m-1}\frac{1}{n+1+k}\right)
    +\frac{1}{m!}\right)\\
    &=
    \frac{1}{m}\left(
    \frac{1}{m-1}\left(-\frac{1}{\prod_{k=1}^{m-1}(n+k)}\right)
    -\frac{1}{m-1}\left(-\frac{1}{\prod_{k=1}^{m-1}(n+1+k)}\right)
    +\frac{1}{m!}\right)\\
    &=
    \frac{1}{m}\left(
    \frac{1}{m-1}
    \left(-\frac{1}{\prod_{k=1}^{m-1}(n+k)}
    +\frac{1}{\prod_{k=1}^{m-1}(n+1+k)}\right)
    +\frac{1}{m!}\right)\\
    &=
    \frac{1}{m}\left(
    \frac{1}{m-1}
    \left(-\frac{1}{\prod_{k=1}^{m-1}(n+k)}
    +\frac{1}{\prod_{k=2}^{m}(n+k)}\right)
    +\frac{1}{m!}\right)\\
    &=
    \frac{1}{m}\left(
    \frac{1}{m-1}
    \left(-\frac{n+m}{\prod_{k=1}^{m}(n+k)}
    +\frac{n+1}{\prod_{k=1}^{m}(n+k)}\right)
    +\frac{1}{m!}\right)\\
    &=
    \frac{1}{m}\left(
    \frac{1}{m-1}\frac{1-m}{\prod_{k=1}^{m}(n+k)}
    +\frac{1}{m!}\right)\\
    &=
    \frac{1}{m}\left(
    -\frac{1}{\prod_{k=1}^{m}(n+k)}
    +\frac{1}{m!}\right)\\
    &=
    \frac{1}{m}\left(
    \frac{1}{m!}-\frac{1}{\prod_{k=1}^{m}(n+k)}
    \right)
  \end{align*}
  であるので,
  $m$に関する数学的帰納法により示せる.
\end{proof**}

\begin{proof*}
  $\forall n \in \ZZ_{>0}$とする.
  $P(m)$を次の命題とする:
  \begin{align*}
   \sum_{i=1}^n\prod_{k=0}^{m}\frac{1}{i+k}=\frac{1}{m}\left(\frac{1}{m!}-\prod_{k=1}^m\frac{1}{n+k}\right).
  \end{align*}
  このとき,
  全ての$m\in\ZZ_{>0}$で$P(m)$が成り立つことを,
  数学的帰納法で示す.

  \paragraph{Base Case:}
  \begin{align*}
    \sum_{i=1}^n\frac{1}{i(i+1)}
    &=\sum_{i=1}^n(\frac{1}{i}-\frac{1}{i+1})\\
    &=\sum_{i=1}^n\frac{1}{i}-\sum_{i=1}^n\frac{1}{i+1}\\
    &=\sum_{i=1}^n\frac{1}{i}-\sum_{i=2}^{n+1}\frac{1}{i}\\
    &=(1+\sum_{i=2}^n\frac{1}{i})-(\frac{1}{n+1}+\sum_{i=2}^{n}\frac{1}{i})\\
    &=1-\frac{1}{n+1}\\
    &=\frac{n}{n+1}.
  \end{align*}
  であるので,
  $P(1)$が成り立つ.
  
  \paragraph{Induction Step:}
  $P(m-1)\implies P(m)$を示す.
  仮定から
  \begin{align*}
    \sum_{i=1}^n\prod_{k=0}^{m-1}\frac{1}{i+k}&=\frac{1}{m-1}\left(\frac{1}{(m-1)!}-\prod_{k=1}^{m-1}\frac{1}{n+k}\right)
\\  
    \sum_{i=1}^{n+1}\prod_{k=0}^{m-1}\frac{1}{i+k}&=\frac{1}{m-1}\left(\frac{1}{(m-1)!}-\prod_{k=1}^{m-1}\frac{1}{n+1+k}\right)
  \end{align*}
  であるので,
  \begin{align*}
    \sum_{i=1}^n\prod_{k=0}^{m}\frac{1}{i+k}
    &=
    \sum_{i=1}^n\frac{1}{\prod_{k=0}^{m}(i+k)}\\
    &=
    \frac{1}{m}\sum_{i=1}^n\frac{m}{\prod_{k=0}^{m}(i+k)}\\
    &=
    \frac{1}{m}\sum_{i=1}^n\frac{m+i-i}{\prod_{k=0}^{m}(i+k)}\\
    &=
    \frac{1}{m}\sum_{i=1}^n\left(\frac{m+i}{\prod_{k=0}^{m}(i+k)}-\frac{i}{\prod_{k=0}^{m}(i+k)}\right)\\
    &=
    \frac{1}{m}\sum_{i=1}^n\left(\frac{1}{\prod_{k=0}^{m-1}(i+k)}-\frac{1}{\prod_{k=1}^{m}(i+k)}\right)\\
    &=
    \frac{1}{m}\sum_{i=1}^n\frac{1}{\prod_{k=0}^{m-1}(i+k)}-\frac{1}{m}\sum_{i=1}^n\frac{1}{\prod_{k=1}^{m}(i+k)}.\\
    &=
    \frac{1}{m}\sum_{i=1}^n\frac{1}{\prod_{k=0}^{m-1}(i+k)}-\frac{1}{m}\sum_{i=2}^{n+1}\frac{1}{\prod_{k=1}^{m}(i-1+k)}\\
    &=
    \frac{1}{m}\sum_{i=1}^n\frac{1}{\prod_{k=0}^{m-1}(i+k)}-\frac{1}{m}\sum_{i=2}^{n+1}\frac{1}{\prod_{k=0}^{m-1}(i+k)}\\
    &=
    \frac{1}{m}\sum_{i=1}^n\frac{1}{\prod_{k=0}^{m-1}(i+k)}-\frac{1}{m}\sum_{i=1}^{n+1}\frac{1}{\prod_{k=0}^{m-1}(i+k)}
    +\frac{1}{m}\frac{1}{\prod_{k=0}^{m-1}(1+k)}\\
    &=
    \frac{1}{m}\left(\sum_{i=1}^n\frac{1}{\prod_{k=0}^{m-1}(i+k)}-\sum_{i=1}^{n+1}\frac{1}{\prod_{k=0}^{m-1}(i+k)}
    +\frac{1}{\prod_{k=0}^{m-1}(1+k)}\right)\\
    &=
    \frac{1}{m}\left(\sum_{i=1}^n\frac{1}{\prod_{k=0}^{m-1}(i+k)}-\sum_{i=1}^{n+1}\frac{1}{\prod_{k=0}^{m-1}(i+k)}
    +\frac{1}{m!}\right)\\
    &=\frac{1}{m}\left(
    \frac{1}{m-1}\left(\frac{1}{(m-1)!}-\prod_{k=1}^{m-1}\frac{1}{n+k}\right)
    -\frac{1}{m-1}\left(\frac{1}{(m-1)!}-\prod_{k=1}^{m-1}\frac{1}{n+1+k}\right)
    +\frac{1}{m!}\right)\\
    &=
    \frac{1}{m}\left(
    \frac{1}{m-1}\left(-\frac{1}{\prod_{k=1}^{m-1}(n+k)}\right)
    -\frac{1}{m-1}\left(-\frac{1}{\prod_{k=1}^{m-1}(n+1+k)}\right)
    +\frac{1}{m!}\right)\\
    &=
    \frac{1}{m}\left(
    \frac{1}{m-1}
    \left(-\frac{1}{\prod_{k=1}^{m-1}(n+k)}
    +\frac{1}{\prod_{k=1}^{m-1}(n+1+k)}\right)
    +\frac{1}{m!}\right)\\
    &=
    \frac{1}{m}\left(
    \frac{1}{m-1}
    \left(-\frac{1}{\prod_{k=1}^{m-1}(n+k)}
    +\frac{1}{\prod_{k=2}^{m}(n+k)}\right)
    +\frac{1}{m!}\right)\\
    &=
    \frac{1}{m}\left(
    \frac{1}{m-1}
    \left(-\frac{n+m}{\prod_{k=1}^{m}(n+k)}
    +\frac{n+1}{\prod_{k=1}^{m}(n+k)}\right)
    +\frac{1}{m!}\right)\\
    &=
    \frac{1}{m}\left(
    \frac{1}{m-1}\frac{1-m}{\prod_{k=1}^{m}(n+k)}
    +\frac{1}{m!}\right)\\
    &=
    \frac{1}{m}\left(
    -\frac{1}{\prod_{k=1}^{m}(n+k)}
    +\frac{1}{m!}\right)\\
    &=
    \frac{1}{m}\left(
    \frac{1}{m!}-\frac{1}{\prod_{k=1}^{m}(n+k)}
    \right)
    .\qedhere
  \end{align*}
\end{proof*}


\begin{prop}
  \label{p:20230729}
  $\forall n\in \ZZ_{>0}$,
  $\sum_{i=1}^n\frac{1}{(2i-1)(2i+1)}=\frac{n}{2n+1}$.
\end{prop}

\begin{proof**}
  $\frac{1}{1\cdot 3}=\frac{1}{3}=\frac{1}{2\cdot 1+1}$である.
  また,
  \begin{align*}
    \sum_{i=1}^n\frac{1}{(2i-1)(2i+1)}-\sum_{i=1}^{n-1}\frac{1}{(2i-1)(2i+1)}
&=\frac{1}{(2n-1)(2n+1)}\\
    \frac{n}{2n+1}-\frac{n-1}{2n-1}
    &=\frac{n(2n-1)-(n-1)(2n+1)}{(2n-1)(2n+1)}\\
    &=\frac{(2n^2-n)-(2n^2-n-1)}{(2n-1)(2n+1)}\\
    &=\frac{1}{(2n-1)(2n+1)}
  \end{align*}
  であるので,
  $n$に関する数学的帰納法により示せる.
\end{proof**}

\begin{proof*}
  $P(n)$を次の命題とする:
  \begin{align*}
    \sum_{i=1}^n\frac{1}{(2i-1)(2i+1)}=\frac{n}{2n+1}.
  \end{align*}
  このとき,
  全ての$n\in\ZZ_{>0}$で$P(n)$が成り立つことを,
  数学的帰納法で示す.

  \paragraph{Base Case:}
  $P(1)$が成り立つことは, 以下から明らか:
  \begin{align*}
    \frac{1}{1\cdot 3}&=\frac{1}{3}.\\
    \frac{1}{2\cdot 1+1}&=\frac{1}{3}.
  \end{align*}

  \paragraph{Induction Step:}
  $P(n-1)\implies P(n)$を示す.
  仮定から$\sum_{i=1}^{n-1}\frac{1}{(2i-1)(2i+1)}=\frac{n-1}{2n-1}$であるので,
  \begin{align*}
    \sum_{i=1}^n\frac{1}{(2i-1)(2i+1)}
    &=\frac{1}{(2n-1)(2n+1)}+\sum_{i=1}^{n-1}\frac{1}{(2i-1)(2i+1)}\\
    &=\frac{1}{(2n-1)(2n+1)}+\frac{n-1}{2n-1}\\
    &=\frac{1+(2n+1)(n-1)}{(2n-1)(2n+1)}\\
    &=\frac{1+(2n^2-n-1)}{(2n-1)(2n+1)}\\
    &=\frac{2n^2-n}{(2n-1)(2n+1)}\\
    &=\frac{n(2n-1)}{(2n-1)(2n+1)}\\
    &=\frac{n}{2n+1}
    .\qedhere
  \end{align*}
\end{proof*}
\begin{rem}
  数学的帰納法を用いず, 以下のように示す方が一般的だと思う:
  \begin{align*}
    \sum_{i=1}^n\frac{1}{(2i-1)(2i+1)}
    &=\frac{1}{2}\sum_{i=1}^n\frac{2}{(2i-1)(2i+1)}\\
    &=\frac{1}{2}\sum_{i=1}^n\frac{(2i+1)-(2i-1)}{(2i-1)(2i+1)}\\
    &=\frac{1}{2}\sum_{i=1}^n\left(\frac{(2i+1)}{(2i-1)(2i+1)}-\frac{(2i-1)}{(2i-1)(2i+1)}\right)\\
    &=\frac{1}{2}\sum_{i=1}^n\left(\frac{1}{2i-1}-\frac{1}{2i+1}\right)\\
    &=\frac{1}{2}\left(\sum_{i=1}^n\frac{1}{2i-1}-\sum_{i=1}^n\frac{1}{2i+1}\right)\\
    &=\frac{1}{2}\left(\sum_{i=0}^{n-1}\frac{1}{2i+1}-\sum_{i=1}^n\frac{1}{2i+1}\right)\\
    &=\frac{1}{2}\left(\frac{1}{2\cdot 0 + 1}+\sum_{i=1}^{n-1}\frac{1}{2i+1}-\frac{1}{2n+1}-\sum_{i=1}^{n-1}\frac{1}{2i+1}\right)\\
    &=\frac{1}{2}\left(1-\frac{1}{2n+1}\right)\\
    &=\frac{1}{2}\frac{2n+1-1}{2n+1}\\
    &=\frac{1}{2}\frac{2n}{2n+1}\\
    &=\frac{n}{2n+1}
    .\qedhere
  \end{align*}
\end{rem}


\begin{prop}
  \label{p:20230730}
  $\forall n\in \ZZ_{>0}$,
  $\sum_{i=1}^n\frac{i}{2^i}=2-\frac{n+2}{2^n}$.
\end{prop}
\begin{proof**}
  $\frac{1}{2^1}=\frac{1}{2}=2-\frac{1+2}{2^1}$である.
  また,
  \begin{align*}
    \sum_{i=1}^n\frac{i}{2^i}-\sum_{i=1}^{n-1}\frac{i}{2^i}
    &=\frac{n}{2^n}\\
    2-\frac{n+2}{2^n}-(2-\frac{n+1}{2^{n-1}}),
    &=-\frac{n+2}{2^n}+\frac{n+1}{2^{n-1}}
    =-\frac{n+2}{2^n}+\frac{2n+2}{2^{n}}
    =\frac{n}{2^{n}}
  \end{align*}
  であるので,
  $n$に関する数学的帰納法により示せる.
\end{proof**}

\begin{proof*}
  $P(n)$を次の命題とする:
  \begin{align*}
    \sum_{i=1}^n\frac{i}{2^i}=2-\frac{n+2}{2^n}.
  \end{align*}
  このとき,
  全ての$n\in\ZZ_{>0}$で$P(n)$が成り立つことを,
  数学的帰納法で示す.

  \paragraph{Base Case:}
  $P(1)$が成り立つことは, 以下から明らか:
  \begin{align*}
    \frac{1}{2^1}&=\frac{1}{2}.\\
    2-\frac{1+2}{2^1}&=2-\frac{3}{2}=\frac{1}{2}.
    \end{align*}

  \paragraph{Induction Step:}
  $P(n-1)\implies P(n)$を示す.
  仮定から$\sum_{i=1}^{n-1}\frac{i}{2^i}=2-\frac{n+1}{2^{n-1}}$であるので,
  \begin{align*}
    \sum_{i=1}^n\frac{i}{2^i}
    &=\frac{n}{2^n}+\sum_{i=1}^{n-1}\frac{i}{2^i}\\
    &=\frac{n}{2^n}+2-\frac{n+1}{2^{n-1}}\\
    &=2+\frac{n}{2^n}-\frac{n+1}{2^{n-1}}\\
    &=2+\frac{n}{2^n}-\frac{2(n+1)}{2^{n}}\\
    &=2-\frac{n+2}{2^n}
    .\qedhere
  \end{align*}
\end{proof*}
\begin{rem}
  \Cref{p:20230721}つまり
  \begin{align*}
    \sum_{i=0}^{n}ix^{i}
    &=x\frac{d}{dx}\sum_{i=0}^{n}x^{i}\\
    &=x\frac{d}{dx}\frac{1-x^n}{1-x}\\
    &=x\frac{nx^{n+1}-(n+1)x^n+1}{(x-1)^2}
  \end{align*}
  をみとめ,
  数学的帰納法を用いず, 以下のように示す方が一般的だと思う:
  \begin{align*}
    \sum_{i=1}^n\frac{i}{2^i}
    &=
    \frac{1}{2}\frac{n\frac{1}{2^{n+1}}-(n+1)\frac{1}{2^n}+1}{(\frac{1}{2}-1)^2}\\
    &=
    \frac{1}{2}\frac{\frac{n}{2^{n+1}}-\frac{(n+1)}{2^n}+1}{\frac{1}{2^2}}\\
    &=
    \frac{2^2}{2}\left(\frac{n}{2^{n+1}}-\frac{(n+1)}{2^n}+1\right)\\
    &=
    \frac{n}{2^{n}}-\frac{2(n+1)}{2^n}+2\\
    &=
    2-\frac{n+2}{2^n}
    .\qedhere
  \end{align*}
\end{rem}


\begin{prop}
  \label{p:20230731}
  $\forall n\in \ZZ_{>0}$,
  $\sum_{i=1}^n \frac{1}{\sqrt{i}+\sqrt{i+1}}=\sqrt{n+1}-1$.
\end{prop}
\begin{proof**}
  $\frac{1}{1+\sqrt{2}}=\frac{(-1+\sqrt{2})}{(1+\sqrt{2})(-1+\sqrt{2})}=(-1+\sqrt{2})$である.
  また,
  \begin{align*}
    \sum_{i=1}^n \frac{1}{\sqrt{i}+\sqrt{i+1}}-\sum_{i=1}^{n-1} \frac{1}{\sqrt{i}+\sqrt{i+1}}
    &=\frac{1}{\sqrt{n}+\sqrt{n+1}}\\
    &=\frac{-\sqrt{n}+\sqrt{n+1}}{(\sqrt{n}+\sqrt{n+1})(-\sqrt{n}+\sqrt{n+1})}\\
    &=\frac{-\sqrt{n}+\sqrt{n+1}}{-n+n+1)}\\
    &=-\sqrt{n}+\sqrt{n+1}\\
    (\sqrt{n+1}-1)-(\sqrt{n}-1)&=\sqrt{n+1}-\sqrt{n}
  \end{align*}
  であるので,
  $n$に関する数学的帰納法により示せる.
\end{proof**}

\begin{proof*}
  $P(n)$を次の命題とする:
  \begin{align*}
    \sum_{i=1}^n \frac{1}{\sqrt{i}+\sqrt{i+1}}=\sqrt{n+1}-1
    .
  \end{align*}
  このとき,
  全ての$n\in\ZZ_{>0}$で$P(n)$が成り立つことを,
  数学的帰納法で示す.

  \paragraph{Base Case:}
  $P(1)$が成り立つことは, 以下から明らか:
  \begin{align*}
    \frac{1}{\sqrt{1}+\sqrt{1+1}}&=\frac{1}{1+\sqrt{2}}=
    \frac{(-1+\sqrt{2})}{(1+\sqrt{2})(-1+\sqrt{2})}=
    -1+\sqrt{2}.\\
    \sqrt{1+1}-1&=\sqrt{2}-1.
    \end{align*}

  \paragraph{Induction Step:}
  $P(n-1)\implies P(n)$を示す.
  仮定から$\sum_{i=1}^{n-1}\frac{1}{\sqrt{i}+\sqrt{i+1}}=\sqrt{n}-1$であるので,
  \begin{align*}
    \sum_{i=1}^n \frac{1}{\sqrt{i}+\sqrt{i+1}}
    &= \frac{1}{\sqrt{n}+\sqrt{n+1}}+\sum_{i=1}^{n-1} \frac{1}{\sqrt{i}+\sqrt{i+1}}\\
    &= \frac{1}{\sqrt{n}+\sqrt{n+1}}+\sqrt{n}-1\\
    &= \frac{-\sqrt{n}+\sqrt{n+1}}{(\sqrt{n}+\sqrt{n+1})(-\sqrt{n}+\sqrt{n+1})}+\sqrt{n}-1\\
    &= \frac{-\sqrt{n}+\sqrt{n+1}}{-n+n+1}+\sqrt{n}-1\\
    &= -\sqrt{n}+\sqrt{n+1}+\sqrt{n}-1\\
    &=\sqrt{n+1}-1
    .\qedhere
  \end{align*}
\end{proof*}
\begin{rem}
  数学的帰納法を用いず, 以下のように示す方が一般的だと思う:
  \begin{align*}
    \sum_{i=1}^n \frac{1}{\sqrt{i}+\sqrt{i+1}}
    &=\sum_{i=1}^n \frac{-\sqrt{i}+\sqrt{i+1}}{(\sqrt{i}+\sqrt{i+1})(-\sqrt{i}+\sqrt{i+1})}\\
    &=\sum_{i=1}^n \frac{-\sqrt{i}+\sqrt{i+1}}{-i+i+1}\\
    &=\sum_{i=1}^n (-\sqrt{i}+\sqrt{i+1})\\
    &=-\sum_{i=1}^n\sqrt{i}+\sum_{i=1}^n\sqrt{i+1})\\
    &=-\sum_{i=1}^n\sqrt{i}+\sum_{i=2}^{n+1}\sqrt{i})\\
    &=-\sqrt{1}-\sum_{i=2}^n\sqrt{i}+\sqrt{n+1}+\sum_{i=2}^{n}\sqrt{i})\\
    &=-\sqrt{1}+\sqrt{n+1}\\
    &=\sqrt{n+1}-1
    .\qedhere
  \end{align*}
\end{rem}

\subsubsection{合同式に関するもの}
\begin{prop}
  \label{p:20230801}
  $\forall n\in \NN$,
  $7^n-2n-1\equiv 0\pmod{4}$.
\end{prop}
\begin{proof**}
  $7^0-2\cdot 0-1=1-0-1=0$である.
  また,
  \begin{align*}
    (7^n-2n-1)=7(7^{n-1}-2(n-1)-1)+12n-8
  \end{align*}
  であるので,
  $n$に関する数学的帰納法により示せる.
\end{proof**}

\begin{proof*}
  $P(n)$を次の命題とする:
  \begin{align*}
    7^n-2n-1\equiv 0\pmod{4}
    .
  \end{align*}
  このとき,
  全ての$n\in\NN$で$P(n)$が成り立つことを,
  数学的帰納法で示す.

  \paragraph{Base Case:}
  $P(0)$が成り立つことは, 以下から明らか:
  \begin{align*}
    7^0-2\cdot 0-1=1-0-1= 0.
  \end{align*}

  \paragraph{Induction Step:}
  $P(n-1)\implies P(n)$を示す.
  仮定から$7^{n-1}-2(n-1)-1\equiv 0\pmod{4}$であるので,
  \begin{align*}
    7^n-2n-1
    &=7(7^{n-1}-2(n-1)-1)+12n-8\\
    &\equiv 7\cdot 0+0n-0\pmod{4}\\
    &=0
    .\qedhere
  \end{align*}
\end{proof*}

\begin{proof**}
  $7^0-2\cdot 0-1=1-0-1=0$,
  $7^1-2\cdot 1-1=7-2-1=4\equiv 0\pmod{4}$
  である.
  また,
  \begin{align*}
    (7^n-2n-1)-(7^{n-2}-2(n-2)-1)
    &=7^{n-2}(7^2-1)-4
    =7^{n-2}48-4
    =4(7^{n-2}12-1)
    \equiv 0 \pmod{4}
  \end{align*}
  であるので,
  $n$に関する数学的帰納法により示せる.
\end{proof**}


\begin{proof*}
  $P(n)$を次の命題とする:
  \begin{align*}
    7^n-2n-1\equiv 0\pmod{4}
    .
  \end{align*}
  また, $P(2n)$を$P'(n)$とし,
  $P(2n+1)$を$P''(n)$とおく.
  このとき,
  全ての$n\in\NN$で$P(n)$が成り立つことを
  示すには,
  全ての$n\in\NN$で$P'(n)$が成り立つことと
  全ての$n\in\NN$で$P''(n)$が成り立つことを示せば良い.

  まず,
  $P(n-2)\implies P(n)$を示す.
  仮定から$7^{n-2}-2(n-2)-1=7^{n-2}-2n+3\equiv 0\pmod{4}$であるので,
  \begin{align*}
    7^n-2n-1
    &=7^2(7^{n-2}-2n+3)+96n-148\\
    &=7^2(7^{n-2}-2n+3)+4(24n-37)\\
    &\equiv 0\pmod{4}
    .\qedhere
  \end{align*}

  
  次に,
  全ての$n\in\NN$で$P'(n)$が成り立つことを,
  数学的帰納法で示す.
  \paragraph{Base Case:}
  $P(0)$が成り立つことは, 以下から明らか:
  \begin{align*}
    7^0-2\cdot 0-1&=1-0-1= 0.
  \end{align*}
    
  \paragraph{Induction Step:}
  $P(n-2)\implies P(n)$はすでに示した.
  したがって,
  $P(2n-2)\implies P(2n)$
  が成り立つ.
  つまり,
  $P'(n-1)\implies P'(n)$
  が成り立つ.

  \paragraph{Base Case:}
  $P(1)$が成り立つことは, 以下から明らか:
  \begin{align*}
    7^1-2\cdot 1-1&=7-2-1=4\equiv 0\pmod{4}.
  \end{align*}
    
  \paragraph{Induction Step:}
  $P(n-2)\implies P(n)$はすでに示した.
  したがって,
  $P(2n-1)\implies P(2n+1)$
  が成り立つ.
  つまり,
  $P''(n-1)\implies P''(n)$
  が成り立つ.
\end{proof*}

\begin{rem}
  数学的帰納法を用いず, 以下のように示す方が一般的だと思う:
  \begin{align*}
    7^n-2n-1&\equiv (-1)^n-2n-1 \pmod{4}
  \end{align*}
  である. $n=2k$のとき,
  \begin{align*}
    (-1)^n-2n-1=(-1)^{2k}-2(2k)-1=1-4k-1=4k \equiv 0 \pmod{4}.
  \end{align*}
  $n=2k-1$のとき,
  \begin{align*}
    (-1)^n-2n-1=(-1)^{2k-1}-2(2k-1)-1=-1-4k+2-1=4k \equiv 0 \pmod{4}.
  \end{align*}
\end{rem}


\begin{rem}
  数学的帰納法を用いず, 以下のように示す方が一般的だと思う:
  $n=2k$のとき,
  \begin{align*}
    7^n-2n-1
    &=7^{2k}-2(2k)-1\\
    &=-2(2k)-1+7^{2k}\\
    &=-4k-1+7^{2k}\\
    &=-4k-1+(8-1)^{2k}\\
    &=-4k-1+\sum_{i=0}^{2k}\binom{2k}{i}(-1)^{2k-i}8^{i}\\
    &=-4k-1+(-1)^{2k}8^{0}+\sum_{i=0}^{2k}\binom{2k}{i}(-1)^{2k-i}8^{i}\\
    &=-4k-1+1+\sum_{i=1}^{2k}\binom{2k}{i}(-1)^{2k-i}8^{i}\\
    &=-4k+\sum_{i=1}^{2k}\binom{2k}{i}(-1)^{2k-i}8^{i}\\
    &=-4k+\sum_{i=0}^{2k-1}\binom{2k}{i}(-1)^{2k-i}8^{i+1}\\
    &=-4k+8\sum_{i=0}^{2k-1}\binom{2k}{i}(-1)^{2k-i}8^{i}\\
    &=4\left(-k+2\sum_{i=0}^{2k-1}\binom{2k}{i}(-1)^{2k-i}8^{i}\right).
  \end{align*}
  $n=2k+1$のとき,
  \begin{align*}
    7^n-2n-1
    &=7^{2k+1}-2(2k+1)-1\\
    &=-2(2k+1)-1+7^{2k+1}\\
    &=-4k-3+(8-1)^{2k+1}\\
    &=-4k-3+\sum_{i=0}^{2k+1}\binom{2k+1}{i}(-1)^{2k+1-i}8^{i}\\
    &=-4k-3+(-1)^{2k+1}8^{0}+\sum_{i=1}^{2k+1}\binom{2k+1}{i}(-1)^{2k+1-i}8^{i}\\
    &=-4k-3-1+\sum_{i=1}^{2k+1}\binom{2k+1}{i}(-1)^{2k+1-i}8^{i}\\
    &=-4k-4+\sum_{i=1}^{2k+1}\binom{2k+1}{i}(-1)^{2k+1-i}8^{i}\\
    &=-4k-4+\sum_{i=1}^{2k+1}\binom{2k+1}{i}(-1)^{2k+1-i}8^{i}\\
    &=-4k-4+\sum_{i=0}^{2k}\binom{2k+1}{i}(-1)^{2k+1-i}8^{i+1}\\
    &=-4k-4+8\sum_{i=0}^{2k}\binom{2k+1}{i}(-1)^{2k+1-i}8^{i}\\
    &=4\left(-k-1+2\sum_{i=0}^{2k}\binom{2k+1}{i}(-1)^{2k+1-i}8^{i}\right).
  \end{align*}
\end{rem}

\begin{prop}
  \label{p:20230802}
  $\forall n\in \NN$,
  $1000^n+(-1)^{n-1}\equiv 0\pmod{7}$.
\end{prop}

\begin{proof**}
  $1000^0+(-1)^{-1}=1-1=0$である.
  また,
  \begin{align*}
    1000^n+(-1)^{n-1}
    &=1000(1000^{n-1}+(-1)^{n-2})+1000(-1)^{n-1}+(-1)^{n-1}\\
    &=1000(1000^{n-1}+(-1)^{n-2})+1001(-1)^{n-1}\\
    &=1000(1000^{n-1}+(-1)^{n-2})+7\cdot 143(-1)^{n-1}
  \end{align*}
  であるので,
  $n$に関する数学的帰納法により示せる.
\end{proof**}

\begin{proof*}
  $P(n)$を次の命題とする:
  \begin{align*}
    1000^n+(-1)^{n-1}\equiv 0\pmod{7}
    .
  \end{align*}
  このとき,
  全ての$n\in\NN$で$P(n)$が成り立つことを,
  数学的帰納法で示す.

  \paragraph{Base Case:}
  $P(0)$が成り立つことは, 以下から明らか:
  \begin{align*}
    1000^0+(-1)^{-1}=1-1=0.
  \end{align*}

  \paragraph{Induction Step:}
  $P(n-1)\implies P(n)$を示す.
  仮定から$1000^{n-1}+(-1)^{n-2}\equiv 0\pmod{7}$であるので,
  \begin{align*}
    1000^n+(-1)^{n-1}
    &=1000(1000^{n-1}+(-1)^{n-2})+1000(-1)^{n-1}+(-1)^{n-1}\\
    &=1000(1000^{n-1}+(-1)^{n-2})+1001(-1)^{n-1}\\
    &=1000(1000^{n-1}+(-1)^{n-2})+7\cdot 143(-1)^{n-1}\\
    &\equiv 1000\cdot 0+0\cdot 143(-1)^{n-1}\pmod{7}\\
    &=0
    .\qedhere
  \end{align*}
\end{proof*}

\begin{rem}
  数学的帰納法を用いず, 以下のように示す方が一般的だと思う:
  \begin{align*}
    1000^n+(-1)^{n-1}&\equiv (-1)^n+(-1)^{n-1}\pmod{7}\\
    &=(-1)^n-(-1)^n\\
    &=0
    .\qedhere
  \end{align*}
\end{rem}

\begin{prop}
  \label{p:20230803}
  $\forall n\in \NN$,
  $3^{3n}-2^n\equiv 0\pmod{25}$.
\end{prop}

\begin{proof**}
  $3^{0}-2^0=1-1=0$である.
  また,
  \begin{align*}
    (3^{3n}-2^n)-3^3(3^{3n-3}-2^{n-1})
    &=(3^{3n}-2^n)-(3^{3n}-3^32^{n-1})\\
    &=-2^n+3^32^{n-1}\\
    &=2^{n-1}(-2+3^3)\\
    &=2^{n-1}(-2+27)\\
    &=2^{n-1}25
  \end{align*}
  であるので,
  $n$に関する数学的帰納法により示せる.
\end{proof**}

\begin{proof*}
  $P(n)$を次の命題とする:
  \begin{align*}
    3^{3n}-2^n\equiv 0\pmod{25}
    .
  \end{align*}
  このとき,
  全ての$n\in\NN$で$P(n)$が成り立つことを,
  数学的帰納法で示す.

  \paragraph{Base Case:}
  $P(0)$が成り立つことは, 以下から明らか:
  \begin{align*}
    3^{0}-2^0=1-1=0.
  \end{align*}

  \paragraph{Induction Step:}
  $P(n-1)\implies P(n)$を示す.
  仮定から$3^{3n-3}-2^{n-1}\equiv 0\pmod{25}$であるので,
  \begin{align*}
    3^{3n}-2^n
    &=3^3(3^{3n-3}-2^{n-1})+3^32^{n-1}-2^n\\
    &=3^3(3^{3n-3}-2^{n-1})+2^{n-1}(3^3-2)\\
    &=3^3(3^{3n-3}-2^{n-1})+2^{n-1}(27-2)\\
    &=3^3(3^{3n-3}-2^{n-1})+2^{n-1}(25)\\
    &\equiv 3^30+2^{n-1}0\pmod{25}\\
    &=0
    .\qedhere
  \end{align*}
\end{proof*}

\begin{rem}
  数学的帰納法を用いず, 以下のように示す方が一般的だと思う:
  \begin{align*}
    3^{3n}-2^n
    =(3^3)^n-2^n
    =27^n-2^n
    =(25+2)^n-2^n
    &\equiv 2^n-2^n \pmod{25}\\
    &=0
    .\qedhere
  \end{align*}
\end{rem}

\begin{prop}
  \label{p:20230804}
  $\forall n\in\NN$,
  $3^{n}-2n+3\equiv 0\pmod{4}$.
\end{prop}

\begin{proof**}
  $3^{0}-2\cdot 0+3=1-0+3=4$である.
  また,
  \begin{align*}
    3^{n}-2n+3
    &=3(3^{n-1}-2(n-1)+3)-3(-2(n-1)+3)-2n+3\\
    &=3(3^{n-1}-2(n-1)+3)+6n-15-2n+3\\
    &=3(3^{n-1}-2(n-1)+3)+4n-12
  \end{align*}
  であるので,
  $n$に関する数学的帰納法により示せる.
\end{proof**}

\begin{proof*}
  $P(n)$を次の命題とする:
  \begin{align*}
    3^{n}-2n+3\equiv 0\pmod{4}
    .
  \end{align*}
  このとき,
  全ての$n\in\NN$で$P(n)$が成り立つことを,
  数学的帰納法で示す.

  \paragraph{Base Case:}
  $P(0)$が成り立つことは, 以下から明らか:
  \begin{align*}
    3^{0}-2\cdot 0+3=1-0+3=4.
  \end{align*}

  \paragraph{Induction Step:}
  $P(n-1)\implies P(n)$を示す.
  仮定から$3^{n-1}-2(n-1)+3\equiv 0\pmod{4}$であるので,
  \begin{align*}
    3^{n}-2n+3
    &=3(3^{n-1}-2(n-1)+3)-3(-2(n-1)+3)-2n+3\\
    &=3(3^{n-1}-2(n-1)+3)+6n-15-2n+3\\
    &=3(3^{n-1}-2(n-1)+3)+4n-12\\
    &=3(3^{n-1}-2(n-1)+3)+4(n-3)\\
    &\equiv 3\cdot 0+0\cdot(n-3)\pmod{4}\\
    &=0
    .\qedhere
  \end{align*}
\end{proof*}

\begin{rem}
  数学的帰納法を用いず, 以下のように示す方が一般的だと思う:
  \begin{align*}
    3^{n}-2n+3\equiv (-1)^n-2n-1\pmod{4}.
  \end{align*}
  $n=2k$のとき,
  \begin{align*}
    (-1)^{2k}-2(2k)-1=1-4k-1=4k\equiv 0 \pmod{4}.
  \end{align*}
  $n=2k+1$のとき,
  \begin{align*}
    (-1)^{2k+1}-2(2k+1)-1=-1-4k-2-1=4k-4=4(k-1)\equiv 0 \pmod{4}.
  \end{align*} 
\end{rem}

\begin{prop}
  \label{p:20230805}
  $\forall n\in\ZZ_{>0}$,
  $3^{3n+1}+7^{2n-1}\equiv 0\pmod{11}$
\end{prop}
\begin{proof**}
  $3^{3+1}+7^{2-1}=81+7=88=8\cdot 11\equiv 0\pmod{11}$である.
  また,
  \begin{align*}
    3^{3n+1}+7^{2n-1}
    &=3^3(3^{3n-2}+7^{2n-3})-7^{2n-3}3^3+7^{2n-1}\\
    &=3^3(3^{3n-2}+7^{2n-3})+7^{2n-3}(-3^3+7^2)\\
    &=3^3(3^{3n-2}+7^{2n-3})+7^{2n-3}(-27+49)\\
    &=3^3(3^{3n-2}+7^{2n-3})+7^{2n-3}22\\
    &=3^3(3^{3n-2}+7^{2n-3})+7^{2n-3}2\cdot 11
  \end{align*}
  であるので,
  $n$に関する数学的帰納法により示せる.
\end{proof**}

\begin{proof*}
  $P(n)$を次の命題とする:
  \begin{align*}
    3^{3n+1}+7^{2n-1}\equiv 0\pmod{11}
    .
  \end{align*}
  このとき,
  全ての$n\in\ZZ_{>0}$で$P(n)$が成り立つことを,
  数学的帰納法で示す.

  \paragraph{Base Case:}
  $P(1)$が成り立つことは, 以下から明らか:
  \begin{align*}
    3^{3+1}+7^{2-1}=81+7=88=8\cdot 11\equiv 0\pmod{11}.
  \end{align*}

  \paragraph{Induction Step:}
  $P(n-1)\implies P(n)$を示す.
  仮定から$3^{3n-2}+7^{2n-3}\equiv 0\pmod{11}$であるので,
  \begin{align*}
    3^{3n+1}+7^{2n-1}
    &=3^3(3^{3n-2}+7^{2n-3})-7^{2n-3}3^3+7^{2n-1}\\
    &=3^3(3^{3n-2}+7^{2n-3})+7^{2n-3}(-3^3+7^2)\\
    &=3^3(3^{3n-2}+7^{2n-3})+7^{2n-3}(-27+49)\\
    &=3^3(3^{3n-2}+7^{2n-3})+7^{2n-3}22\\
    &=3^3(3^{3n-2}+7^{2n-3})+7^{2n-3}2\cdot 11\\
    &\equiv 3^3\cdot 0+7^{2n-3}2\cdot 0\pmod{11}\\
    &=0
    .\qedhere
  \end{align*}
\end{proof*}

\begin{rem}
  数学的帰納法を用いず, 以下のように示す方が一般的だと思う:
  \begin{align*}
    3^{3n+1}+7^{2n-1}
    &=3^{3n}3+7^{2n-2}7\\
    &=27^{n}3+49^{n-1}7\\
    &\equiv 5^n3+5^{n-1}7\pmod{11}\\
    &=5^{n-1}(5\cdot 3 + 7)\\
    &=5^{n-1}22\\
    &=5^{n-1}2\cdot 11\\
    &\equiv 0\pmod{11}
    .\qedhere
  \end{align*}
\end{rem}

\begin{rem}
  $\ZZ/11\ZZ$は体であり,
  $7^{-1}$が存在する.
  この意味で, この命題は$n=0$のときも成り立つ.
\end{rem}

\begin{prop}
  \label{p:20230807}
  $\forall n\in\ZZ_{>0}$,
  $4\cdot 3^{2n-1}+2^{4n}\equiv 0\pmod{28}$
\end{prop}

\begin{proof**}
  $4\cdot 3^{1}+2^{4}=12+16=28\equiv 0\pmod{28}$である.
  また,
  \begin{align*}
    4\cdot 3^{2n-1}+2^{4n}
    &=3^2(4\cdot 3^{2n-3}+2^{4n-4})
    -3^22^{4n-4}+2^{4n}\\
    &=3^2(4\cdot 3^{2n-3}+2^{4n-4})
    2^{4n-4}(-3^2+2^{4})\\
    &=3^2(4\cdot 3^{2n-3}+2^{4n-4})
    2^{4n-4}(-9+16)\\
    &=3^2(4\cdot 3^{2n-3}+2^{4n-4})
    -2^{4n-4}7
  \end{align*}
  であるので,
  $n$に関する数学的帰納法により示せる.
\end{proof**}

\begin{proof*}
  $P(n)$を次の命題とする:
  \begin{align*}
    4\cdot 3^{2n-1}+2^{4n}\equiv 0\pmod{28}
    .
  \end{align*}
  このとき,
  全ての$n\in\ZZ_{>0}$で$P(n)$が成り立つことを,
  数学的帰納法で示す.

  \paragraph{Base Case:}
  $P(1)$が成り立つことは, 以下から明らか:
  \begin{align*}
    4\cdot 3^{2-1}+2^{4}=12+16\equiv 0\pmod{28}
  \end{align*}

  \paragraph{Induction Step:}
  $n>1$とする.
  $P(n-1)\implies P(n)$を示す.
  仮定から$4\cdot 3^{2n-3}+2^{4n-4}\equiv 0\pmod{28}$であるので,
  \begin{align*}
    4\cdot 3^{2n-1}+2^{4n}
    &=3^2(4\cdot 3^{2n-3}+2^{4n-4})
    -3^22^{4n-4}+2^{4n}\\
    &=3^2(4\cdot 3^{2n-3}+2^{4n-4})
    +2^{4n-4}(-3^2+2^{4})\\
    &=3^2(4\cdot 3^{2n-3}+2^{4n-4})
    +2^{4n-4}(-9+16)\\
    &=3^2(4\cdot 3^{2n-3}+2^{4n-4})
    -2^{4n-4}7.
  \end{align*}
  $n>1$であるので, $4n-4\geq 4$である.
  したがって,
  \begin{align*}
    4\cdot 3^{2n-1}+2^{4n}
    &=3^2(4\cdot 3^{2n-3}+2^{4n-4})
    -2^{4n-4}7\\
    &\equiv 3^2\cdot 0-0\pmod{28}\\
    &=0
    .\qedhere
  \end{align*}
\end{proof*}

\begin{rem}
  数学的帰納法を用いず, 以下のように示す方が一般的だと思う:
  \begin{align*}
    9^{n-1}3+16^{n-1}4
    &\equiv 2^{n-1}3+2^{n-1}4 \pmod{7}\\
    &= 2^{n-1}(3+4)\\
    &= 2^{n-1}7\\
    &\equiv 0\pmod{7}.
  \end{align*}
  であるので,
  \begin{align*}
    4\cdot 3^{2n-1}+2^{4n}
    &=4(3^{2n-1}+2^{4n-2})\\
    &=4(9^{n-1}3+16^{n-1}4)\\
    &\equiv 0 \pmod{28}.\qedhere
  \end{align*}
\end{rem}

\subsection{不等式に関するもの}
\subsubsection{冪に関するもの}
\begin{prop}
  \label{p:20230808}
  $\forall n\in\NN$, $2^n> n$.
\end{prop}
\begin{proof**}
  $2^0=1> 0$である.
  また,
  $n=(n-1)+1$であるが,
  $n\geq 1$に対し,
  $2^{n-1}+1\leq  2^{n-1}+2^{n-1}=2^n$であるので,
  $n$に関する数学的帰納法により示せる.
\end{proof**}
\begin{proof*}
  $P(n)$を次の命題とする:
  \begin{align*}
    2^n>n
    .
  \end{align*}
  このとき,
  全ての$n\in\NN$で$P(n)$が成り立つことを,
  数学的帰納法で示す.

  \paragraph{Base Case:}
  $P(0)$が成り立つことは, 以下から明らか:
  \begin{align*}
    2^0=1> 0.
  \end{align*}
  \paragraph{Induction Step:}
  $n\geq 1$とする.
  $P(n-1)\implies P(n)$を示す.
  仮定から$2^{n-1}>n-1$である.
  また, $n\geq 1$であるので$2^{n-1}\geq 1$であるので,
  \begin{align*}
    n=(n-1)+1<2^{n-1}+1\geq 2^{n-1}+2^{n-1}=2^{n-1}(1+1)=2^n
    .\qedhere
  \end{align*}
\end{proof*}

\begin{rem}
  数学的帰納法を用いず, 以下のように示す方が一般的だと思う:
  \begin{align*}
    f(x)=2^x-x
  \end{align*}
  とおく.
  \begin{align*}
    \frac{d}{dx}f(x)=\log 2 \cdot 2^x-1
  \end{align*}
  であるので, $x\geq 0$において, $\frac{d}{dx}f(x)> 0$.
  したがって, $f(x)$は$x\geq 0$において単調増加.
  また$f(0)=1>0$であるので,
  $x\geq 0$において$f(x)\geq 0$.
  よって, $x\geq 0$において$2^x>x$.
\end{rem}

\begin{rem}
  数学的帰納法を用いず, 以下のように示すこともできる:
  $2^0=1>0$である.
  また, $n\geq 1$に対し,
  \begin{align*}
    2^n=(1+1)^n=\sum_{i=0}^n \binom{n}{i}\geq \binom{n}{0}+\binom{n}{1}=1+n>n
    .\qedhere
  \end{align*}
\end{rem}

\begin{prop}
  \label{p:20230809}
  $n\in\NN, n\geq 2\implies 2^n \geq n+2$.
\end{prop}
\begin{proof**}
  $2^2=4=2+2$である.
  また,
  $2^n=2\cdot 2^{n-1}=2^{n-1}+2^{n-1}$であるが,
  $n\geq 2$に対し,
  $2^{n-1} > 1$であるので,
  $n$に関する数学的帰納法により示せる.
\end{proof**}
\begin{proof*}
  $P(n)$を次の命題とする:
  \begin{align*}
    2^n \geq n+2
    .
  \end{align*}
  このとき,
  $2$以上の整数$n$で$P(n)$が成り立つことを,
  数学的帰納法で示す.

  \paragraph{Base Case:}
  $P(2)$が成り立つことは, 以下から明らか:
  \begin{align*}
    2^2=4=2+2.
  \end{align*}
  \paragraph{Induction Step:}
  $n\geq 3$とする.
  $P(n-1)\implies P(n)$を示す.
  仮定から$2^{n-1}\geq n+1$である.
  また, $n\geq 3$であるので$2^{n-1}\geq 1$であるので,
  \begin{align*}
    2^n=2^{n-1}(1+1)=2^{n-1}+2^{n-1}\geq n+1+2^{n-1}\geq n+1+1=n+2
    .\qedhere
  \end{align*}
\end{proof*}

\begin{rem}
  数学的帰納法を用いず, 以下のように示す方が一般的だと思う:
  \begin{align*}
    f(x)=2^x-x-2
  \end{align*}
  とおく.
  \begin{align*}
    \frac{d}{dx}f(x)=\log 2 \cdot 2^x-1
  \end{align*}
  であるので, $x\geq 0$において, $\frac{d}{dx}f(x)> 0$.
  したがって, $f(x)$は$x\geq 0$において単調増加.
  また$f(2)=0$であるので,
  $x\geq 2$において$f(x)\geq 0$.
  よって, $x\geq 2$において$2^x>x+2$.
\end{rem}

\begin{rem}
  数学的帰納法を用いず, 以下のように示すこともできる:
  $n\geq 2$に対し,
  \begin{align*}
    2^n=(1+1)^n=\sum_{i=0}^n \binom{n}{i}\geq \binom{n}{0}+\binom{n}{1}+\binom{n}{n}=1+n+1=n+2
    .\qedhere
  \end{align*}
\end{rem}

\begin{prop}
  \label{p:20230810}
  $n\in\NN, n\geq 3 \implies 2^n\geq n+5$.
\end{prop}
\begin{proof**}
  $2^3=8=3+5$である.
  また,
  $2^n=2\cdot 2^{n-1}=2^{n-1}+2^{n-1}$であるが,
  $n\geq 2$に対し,
  $2^{n-1} > 1$であるので,
  $n$に関する数学的帰納法により示せる.
\end{proof**}
\begin{proof*}
  $P(n)$を次の命題とする:
  \begin{align*}
    2^n\geq n+5
    .
  \end{align*}
  このとき,
  $3$以上の整数$n$で$P(n)$が成り立つことを,
  数学的帰納法で示す.

  \paragraph{Base Case:}
  $P(3)$が成り立つことは, 以下から明らか:
  \begin{align*}
    2^3=8=3+5.
  \end{align*}
  \paragraph{Induction Step:}
  $n\geq 4$とする.
  $P(n-1)\implies P(n)$を示す.
  仮定から$2^{n-1}\geq n+4$である.
  また, $n\geq 1$であるので$2^{n-1}\geq 1$であるので,
  \begin{align*}
    2^n=2^{n-1}(1+1)=2^{n-1}+2^{n-1}\geq n+4+2^{n-1}\geq n+4+1=n+5
    .\qedhere
  \end{align*}
\end{proof*}

\begin{rem}
  数学的帰納法を用いず, 以下のように示す方が一般的だと思う:
  \begin{align*}
    f(x)=2^x-x-5
  \end{align*}
  とおく.
  \begin{align*}
    \frac{d}{dx}f(x)=\log 2 \cdot 2^x-1
  \end{align*}
  であるので, $x\geq 0$において, $\frac{d}{dx}f(x)> 0$.
  したがって, $f(x)$は$x\geq 0$において単調増加.
  また$f(3)=0$であるので,
  $x\geq 3$において$f(x)\geq 0$.
  よって, $x\geq 3$において$2^x>x+3$.
\end{rem}

\begin{rem}
  数学的帰納法を用いず, 以下のように示すこともできる:
  $n\geq 3$に対し, $n+2\geq 5$であるから,
  \begin{align*}
    2^n=(1+1)^n=\sum_{i=0}^n \binom{n}{i}\geq \binom{n}{0}+\binom{n}{1}+\binom{n}{n-1}+\binom{n}{n}=1+n+n+1=n+(n+2)\geq n+5
    .\qedhere
  \end{align*}
\end{rem}

\begin{prop}
  \label{p:20230813}
  $n\in\NN, n\geq 3\implies 2^n> 2n+1$.
\end{prop}
\begin{proof**}
  $2^3=8>7=6+1$である.
  また,
  $2^n=2\cdot 2^{n-1}=2^{n-1}+2^{n-1}$であるが,
  $n\geq 2$に対し,
  $2^{n-1} \geq 2$であるので,
  $n$に関する数学的帰納法により示せる.
\end{proof**}
\begin{proof*}
  $P(n)$を次の命題とする:
  \begin{align*}
    2^n> 2n+1
    .
  \end{align*}
  このとき,
  $3$以上の整数$n$で$P(n)$が成り立つことを,
  数学的帰納法で示す.

  \paragraph{Base Case:}
  $P(3)$が成り立つことは, 以下から明らか:
  \begin{align*}
    2^3=8>7=6+1.
  \end{align*}
  \paragraph{Induction Step:}
  $n\geq 4$とする.
  $P(n-1)\implies P(n)$を示す.
  仮定から$2^{n-1}\geq 2n-1$である.
  また, $n\geq 2$であるので$2^{n-1}\geq 2$であるので,
  \begin{align*}
    2^n=2^{n-1}(1+1)=2^{n-1}+2^{n-1}\geq 2n-1+2^{n-1}\geq 2n-1+2=2n+1
    .\qedhere
  \end{align*}
\end{proof*}

\begin{rem}
  数学的帰納法を用いず, 以下のように示す方が一般的だと思う:
  \begin{align*}
    f(x)=2^x-2x-1
  \end{align*}
  とおく.
  \begin{align*}
    \frac{d}{dx}f(x)=\log 2 \cdot 2^x-2
  \end{align*}
  であるので, $x\geq 1$において, $\frac{d}{dx}f(x)> 0$.
  したがって, $f(x)$は$x\geq 1$において単調増加.
  また$f(3)=8-6-1=1$であるので,
  $x\geq 3$において$f(x) > 0$.
  よって, $x\geq 3$において$2^x>2x+1$.
\end{rem}

\begin{rem}
  数学的帰納法を用いず, 以下のように示すこともできる:
  $n\geq 3$に対し, 
  \begin{align*}
    2^n=(1+1)^n=\sum_{i=0}^n \binom{n}{i}\geq \binom{n}{0}+\binom{n}{1}+\binom{n}{n-1}+\binom{n}{n}=1+n+n+1=2n+2> 2n+1
    .\qedhere
  \end{align*}
\end{rem}

\begin{prop}
  \label{p:20230814}
  $n\in\NN, n\geq 4\implies 2^n > 3n$.
\end{prop}
\begin{proof**}
  $2^4=16>12=3\cdot 4$である.
  また,
  $2^n=2\cdot 2^{n-1}=2^{n-1}+2^{n-1}$であるが,
  $n\geq 3$に対し,
  $2^{n-1} \geq 3$であるので,
  $n$に関する数学的帰納法により示せる.
\end{proof**}
\begin{proof*}
  $P(n)$を次の命題とする:
  \begin{align*}
    2^n > 3n
    .
  \end{align*}
  このとき,
  $4$以上の整数$n$で$P(n)$が成り立つことを,
  数学的帰納法で示す.

  \paragraph{Base Case:}
  $P(4)$が成り立つことは, 以下から明らか:
  \begin{align*}
    2^4=16>12=3\cdot 4.
  \end{align*}
  \paragraph{Induction Step:}
  $n\geq 5$とする.
  $P(n-1)\implies P(n)$を示す.
  仮定から$2^{n-1}\geq 3n-3$である.
  また, $n\geq 3$であるので$2^{n-1}\geq 4$であるので,
  \begin{align*}
    2^n=2^{n-1}(1+1)=2^{n-1}+2^{n-1}\geq 3n-3+2^{n-1}\geq 3n-3+4=3n+1>3n
    .\qedhere
  \end{align*}
\end{proof*}

\begin{rem}
  数学的帰納法を用いず, 以下のように示す方が一般的だと思う:
  \begin{align*}
    f(x)=2^x-3x
  \end{align*}
  とおく.
  \begin{align*}
    \frac{d}{dx}f(x)=\log 2 \cdot 2^x-3
  \end{align*}
  であるので, $x\geq 2$において, $\frac{d}{dx}f(x)> 0$.
  したがって, $f(x)$は$x\geq 2$において単調増加.
  また$f(4)=16-12=4$であるので,
  $x\geq 4$において$f(x) > 0$.
  よって, $x\geq 4$において$2^x>3x$.
\end{rem}

\begin{rem}
  数学的帰納法を用いず, 以下のように示すこともできる:
  $n\geq 4$に対し, $\frac{n+2}{2}\geq 3$
  \begin{align*}
    2^n=(1+1)^n=\sum_{i=0}^n \binom{n}{i}
    &> \binom{n}{1}+\binom{n}{2}+\binom{n}{n-1}\\
    &=n+\frac{n(n-1)}{2}+n\\
    &=\left(\frac{n+3}{2}\right)n\\
    &=\left(\frac{n+2}{2}+\frac{1}{2}\right)n
    > 3n
    .\qedhere
  \end{align*}
\end{rem}


\begin{prop}
  \label{p:20230815}
  $n\in\NN, n\geq 4\implies 2^n\geq n^2$.
\end{prop}
\begin{proof**}
  $2^4=16\geq 16$である.
  また,
  $2^n=2\cdot 2^{n-1}$であるが,
  $n\geq 4$に対し,
  $2(n-1)^2-n^2=n^2-4n+1=(n-2)^2-3\geq 1>0$であるので,
  $n$に関する数学的帰納法により示せる.
\end{proof**}
\begin{proof*}
  $P(n)$を次の命題とする:
  \begin{align*}
    2^n\geq n^2
    .
  \end{align*}
  このとき,
  $4$以上の整数$n$で$P(n)$が成り立つことを,
  数学的帰納法で示す.

  \paragraph{Base Case:}
  $P(4)$が成り立つことは, 以下から明らか:
  \begin{align*}
    2^4=16=4^2.
  \end{align*}
  \paragraph{Induction Step:}
  $n\geq 5$とする.
  $P(n-1)\implies P(n)$を示す.
  仮定から$2^{n-1}\geq (n-1)^2$である.
  また,
  \begin{align*}
    2(n-1)^2-n^2=n^2-4n+1=(n-2)^2-3
  \end{align*}
  であり,
  $n\geq 3$であるので,
  $2(n-1)^2-n^2>0$, つまり
  \begin{align*}
    2(n-1)^2>n^2.
  \end{align*}
  よって,
  \begin{align*}
    2^n=2\cdot 2^{n-1}\geq 2(n-1)^2>n^2
    .\qedhere
  \end{align*}
\end{proof*}

\begin{rem}
  数学的帰納法を用いず, 以下のように示す方が一般的だと思う:
  \begin{align*}
    f(x)=2^x-x^2
  \end{align*}
  とおく.
  \begin{align*}
    \frac{d}{dx}f(x)&=\log 2 \cdot 2^x-2x\\
    \frac{d^2}{dx^2}f(x)&=(\log 2)^2 \cdot 2^x-2\\
  \end{align*}
  であるので, $x\geq 1$において,
  $\frac{d^2}{dx^2}f(x)>0$.
  また, $\frac{d}{dx}f(1)>0$であるので,
  $x\geq 1$において,
  $\frac{d}{dx}f(x)> 0$.
  したがって, $f(x)$は$x\geq 1$において単調増加.
  また$f(4)=16-16=0$であるので,
  $x\geq 4$において$f(x) \geq 0$.
  よって, $x\geq 4$において$2^x\geq x^2$.
\end{rem}

\begin{rem}
  数学的帰納法を用いず, 以下のように示すこともできる:
  $n=4$に対し, $2^4=16=4^2$.
  $n\geq 5$に対し, 
  \begin{align*}
    2^n=(1+1)^n=\sum_{i=0}^n \binom{n}{i}
    &> \binom{n}{1}+\binom{n}{2}+\binom{n}{n-2}
    =n+\frac{n(n-1)}{2}+\frac{n(n-1)}{2}
    =n^2
    .\qedhere
  \end{align*}
\end{rem}


\begin{prop}
  \label{p:20230816}
$n\in\NN, n\geq 5\implies 2^n> n^2-2n+15$.
\end{prop}
\begin{proof**}
  $n^2-2n+15=(n-1)^2+14$である.
  $2^5=32 > 30=4^2+14$である.
  また,
  $2^n=2\cdot 2^{n-1}$であるが,
  $n\geq 6$に対し,
  $2((n-2)^2+14)-((n-1)^2+14)=2(n^2-4n+4)-(n^2-2n+1)+14=n^2-6n+21=(n-3)^2+12>0$
  であるので,
  $n$に関する数学的帰納法により示せる.
\end{proof**}
\begin{proof*}
  $P(n)$を次の命題とする:
  \begin{align*}
    2^n > n^2-2n+15
    .
  \end{align*}
  このとき,
  $5$以上の整数$n$で$P(n)$が成り立つことを,
  数学的帰納法で示す.

  \paragraph{Base Case:}
  $P(5)$が成り立つことは, 以下から明らか:
  \begin{align*}
    2^5&=32\\
    5^2-2\cdot 5+15&=30.
  \end{align*}
  \paragraph{Induction Step:}
  $n\geq 6$とする.
  $P(n-1)\implies P(n)$を示す.
  仮定から$2^{n-1} > (n-1)^2-2(n-1)+15$である.
  また,
  \begin{align*}
    2((n-1)^2-2(n-1)+15)-(n^2-2n+15)=2(n^2-4n+4)-(n^2-2n+1)+14=n^2-6n+21=(n-3)^2+12>0
  \end{align*}
  であるので,
  \begin{align*}
    2((n-1)^2-2(n-1)+15)>n^2-2n+15.
  \end{align*}
  よって,
  \begin{align*}
    2^n=2\cdot 2^{n-1}>2((n-1)^2-2(n-1)+15)>n^2-2n+15
    .\qedhere
  \end{align*}
\end{proof*}

\begin{rem}
  数学的帰納法を用いず, 以下のように示す方が一般的だと思う:
  \begin{align*}
    f(x)=2^x-(x^2-2x+15)
  \end{align*}
  とおく.
  \begin{align*}
    \frac{d}{dx}f(x)&=\log 2 \cdot 2^x-2x+2\\
    \frac{d^2}{dx^2}f(x)&=(\log 2)^2 \cdot 2^x-2\\
  \end{align*}
  であるので, $x\geq 1$において,
  $\frac{d^2}{dx^2}f(x)>0$.
  また, $\frac{d}{dx}f(1)=2\log 2>0$であるので,
  $x\geq 1$において,
  $\frac{d}{dx}f(x)> 0$.
  したがって, $f(x)$は$x\geq 1$において単調増加.
  また$f(5)=32-25+10-15=2>0$であるので,
  $x\geq 5$において$f(x) > 0$.
  よって, $x\geq 5$において$2^x > x^2$.
\end{rem}

\begin{rem}
  数学的帰納法を用いず, 以下のように示すこともできる:
  $n\geq 5$に対し, $n-(-2n+15)=3n-15\geq 0$であるので,
  $n>-2n+15$. よって,
  \begin{align*}
    2^n=(1+1)^n=\sum_{i=0}^n \binom{n}{i}
    &> \binom{n}{1}+\binom{n}{2}+\binom{n}{n-2}+\binom{n}{n-1}\\
    &=n+\frac{n(n-1)}{2}+\frac{n(n-1)}{2}+n\\
    &=n^2+n\\
    &>n^2-2n+5
    .\qedhere
  \end{align*}
\end{rem}

\begin{prop}
  \label{p:20230817}
  $n\in\NN, n\geq 10\implies 2^n>10n^2$.
\end{prop}
\begin{proof**}
  $2^{10}=1024>1000=10n^2$である.
  また,
  $2^n=2\cdot 2^{n-1}$であるが,
  $n\geq 6$に対し,
  $2\cdot 10(n-1)^2-10n^2=10(n^2-4n+2)=10((n-2)^2-4)\geq 0$
  であるので,
  $n$に関する数学的帰納法により示せる.
\end{proof**}
\begin{proof*}
  $P(n)$を次の命題とする:
  \begin{align*}
    2^n > 10n^2
    .
  \end{align*}
  このとき,
  $10$以上の整数$n$で$P(n)$が成り立つことを,
  数学的帰納法で示す.

  \paragraph{Base Case:}
  $P(10)$が成り立つことは, 以下から明らか:
  \begin{align*}
    2^{10}&=1024\\
    10\cdot 10^2&=1000
  \end{align*}
  \paragraph{Induction Step:}
  $n\geq 11$とする.
  $P(n-1)\implies P(n)$を示す.
  仮定から$2^{n-1} > 10(n-1)^2$である.
  また,
  \begin{align*}
    2\cdot 10(n-1)^2-10n^2=10(n^2-4n+2)=10((n-2)^2-4)\geq 0
  \end{align*}
  であるので,
  \begin{align*}
    2\cdot 10(n-1)^2>10n^2.
  \end{align*}
  よって,
  \begin{align*}
    2^n=2\cdot 2^{n-1}>2(10(n-1)^2)>10n^2
    .\qedhere
  \end{align*}
\end{proof*}

\begin{rem}
  数学的帰納法を用いず, 以下のように示す方が一般的だと思う:
  \begin{align*}
    f(x)=2^x-10x^2
  \end{align*}
  とおく.
  \begin{align*}
    \frac{d}{dx}f(x)&=\log 2 \cdot 2^x-10x\\
    \frac{d^2}{dx^2}f(x)&=(\log 2)^2 \cdot 2^x-10\\
  \end{align*}
  であるので, $x\geq 4$において,
  $\frac{d^2}{dx^2}f(x)>0$.
  また, $\frac{d}{dx}f(6)>0$であるので,
  $x\geq 6$において,
  $\frac{d}{dx}f(x)> 0$.
  したがって, $f(x)$は$x\geq 6$において単調増加.
  また$f(10)=1024-1000=24>0$であるので,
  $x\geq 10$において$f(x) > 0$.
  よって, $x\geq 10$において$2^x > 10x^2$.
\end{rem}

\begin{prop}
  \label{p:20230818}
  $n\in\NN, n\geq 10\implies 2^n\geq n^3$.
\end{prop}
\begin{proof**}
  $2^{10}=1024>1000=n^3$である.
  また, $2\cdot(11-1)^3=2000$, $11^3=1331$であることから,
  $n\geq 11$に対し,
  $2\cdot (n-1)^3>n^3$
  である.
  したがって,
  $2^n=2\cdot 2^{n-1}$であることから,
  $n$に関する数学的帰納法により示せる.
\end{proof**}
\begin{proof*}
  $P(n)$を次の命題とする:
  \begin{align*}
    2^n\geq n^3
    .
  \end{align*}
  このとき,
  $10$以上の整数$n$で$P(n)$が成り立つことを,
  数学的帰納法で示す.

  \paragraph{Base Case:}
  $P(10)$が成り立つことは, 以下から明らか:
  \begin{align*}
    2^{10}&=1024,\\
    10^3&=1000.
  \end{align*}
  \paragraph{Induction Step:}
  $n\geq 11$とする.
  $P(n-1)\implies P(n)$を示す.
  仮定から$2^{n-1} > 10(n-1)^2$である.
  また, $2\cdot(11-1)^3=2000$, $11^3=1331$であることから,
  $n\geq 11$に対し,
  \begin{align*}
    2\cdot (n-1)^3>n^3
  \end{align*}
  であることがわかる.
  よって,
  \begin{align*}
    2^n=2\cdot 2^{n-1}>2(n-1)^3>n^3
    .\qedhere
  \end{align*}
\end{proof*}

\begin{rem}
  数学的帰納法を用いず, 以下のように示す方が一般的だと思う:
  \begin{align*}
    f(x)=2^x-x^3
  \end{align*}
  とおく.
  \begin{align*}
    \frac{d}{dx}f(x)&=\log 2 \cdot 2^x-3x^2\\
    \frac{d^2}{dx^2}f(x)&=(\log 2)^2 \cdot 2^x-6x\\
    \frac{d^3}{dx^3}f(x)&=(\log 2)^3 \cdot 2^x-6\\
  \end{align*}
  であるので,
  $x\geq 3$において,
  $\frac{d^3}{dx^3}f(x)>0$.
  よって, 
  $\frac{d^2}{dx^2}f(x)$は,
  $x\geq 3$において単調増加.
  また,
  $\frac{d^2}{dx^2}f(5)>0$であるから,
  $x\geq 5$において,
  $\frac{d^2}{dx^2}f(x)>0$.
  よって, 
  $\frac{d}{dx}f(x)$は,
  $x\geq 5$において単調増加.  
  また, $\frac{d}{dx}f(8)>0$であるので,
  $x\geq 8$において,
  $\frac{d}{dx}f(x)> 0$.
  したがって, $f(x)$は$x\geq 8$において単調増加.
  また$f(10)=1024-1000=24>0$であるので,
  $x\geq 10$において$f(x) > 0$.
  よって, $x\geq 10$において$2^x > x^3$.
\end{rem}


\begin{prop}
  \label{p:20230819}
  $n\in\NN \implies 3^n \geq 2n+1$.
\end{prop}
\begin{proof**}
  $3^{0}=1= 2\cdot 0+1$である.
  また, 
  $n\geq 1$に対し$3^{n-1}\geq 1$であり,
  $3^n=3^{n-1}+2\cdot 3^{n-1}$であることから,
  $n$に関する数学的帰納法により示せる.
\end{proof**}
\begin{proof*}
  $P(n)$を次の命題とする:
  \begin{align*}
    3^n \geq 2n+1
    .
  \end{align*}
  このとき,
  すべての$n\in\NN$で$P(n)$が成り立つことを,
  数学的帰納法で示す.

  \paragraph{Base Case:}
  $P(0)$が成り立つことは, 以下から明らか:
  \begin{align*}
    3^{0}&=1\\
     2\cdot 0+1&=1.
  \end{align*}
  \paragraph{Induction Step:}
  $P(n-1)\implies P(n)$を示す.
  仮定から$3^{n-1} \geq 2n-1$である.
  また, 
  $n\geq 1$に対し, $3^n\geq 1$であるので,
  \begin{align*}
    3^n=3\cdot 3^{n-1}=(1+2)3^{n-1}=3^{n-1}+2\cdot 3^{n-1}\geq 2n-1+2\cdot 3^{n-1}\geq 2n-1+2=2n+1
    .\qedhere
  \end{align*}
\end{proof*}

\begin{rem}
  数学的帰納法を用いず, 以下のように示す方が一般的だと思う:
  \begin{align*}
    f(x)=3^x-(2x+1)
  \end{align*}
  とおく.
  \begin{align*}
    \frac{d}{dx}f(x)&=\log 3 \cdot 3^x-2
  \end{align*}
  であるので,
  $x\geq 1$において,
  $\frac{d}{dx}f(x)> 0$.
  したがって, $f(x)$は$x\geq 1$において単調増加.
  また$f(1)=3-3=0$であるので,
  $x\geq 1$において$f(x) \geq 0$.
  よって, $x\geq 1$において$3^x \geq 2x+1$.
  また, $f(0)=3^0-(2\cdot0+1)=0$であるので, $x=0$のときも,
  $3^x\geq 2x+1$.
\end{rem}
\begin{rem}
  数学的帰納法を用いず, 以下のように示すこともできる:
  $n=0$に対し, $3^0=1=2\cdot0+1$.
  $n\geq 1$に対し, 
  \begin{align*}
    3^n=(1+2)^n=\sum_{i=0}^n \binom{n}{i}2^i
    &\geq \binom{n}{0}+\binom{n}{1}2
    =1+2n
    .\qedhere
  \end{align*}
\end{rem}

\begin{prop}
  \label{p:20230820}
  $n\in\NN,n\geq 3\implies 3^n> 4n+10$.
\end{prop}
\begin{proof**}
  $3^{3}=27>22= 4\cdot 3+10$である.
  また, 
  $n\geq 2$に対し$3^{n-1}\geq 2$であり,
  $3^n=3^{n-1}+2\cdot 3^{n-1}$であることから,
  $n$に関する数学的帰納法により示せる.
\end{proof**}
\begin{proof*}
  $P(n)$を次の命題とする:
  \begin{align*}
    3^n> 4n+10
    .
  \end{align*}
  このとき,
  $3$以上の整数$n$で$P(n)$が成り立つことを,
  数学的帰納法で示す.

  \paragraph{Base Case:}
  $P(3)$が成り立つことは, 以下から明らか:
  \begin{align*}
    3^{3}&=27\\
     4\cdot 3+10&=22.
  \end{align*}
  \paragraph{Induction Step:}
  $n\geq 4$とし,
  $P(n-1)\implies P(n)$を示す.
  仮定から$3^{n-1}> 4n+6$である.
  また, 
  $n\geq 2$に対し, $3^{n-1}\geq 2$であるので,
  \begin{align*}
    3^n=3\cdot 3^{n-1}=(1+2)3^{n-1}=3^{n-1}+2\cdot 3^{n-1}> 4n+6+2\cdot 3^{n-1}\geq 4n+6+2\cdot 2=4n+10
    .\qedhere
  \end{align*}
\end{proof*}

\begin{rem}
  数学的帰納法を用いず, 以下のように示す方が一般的だと思う:
  \begin{align*}
    f(x)=3^x-(4x+10)
  \end{align*}
  とおく.
  \begin{align*}
    \frac{d}{dx}f(x)&=\log 3 \cdot 3^x-4
  \end{align*}
  であるので,
  $x\geq 2$において,
  $\frac{d}{dx}f(x)> 0$.
  したがって, $f(x)$は$x\geq 2$において単調増加.
  また$f(3)=27-24=3>0$であるので,
  $x\geq 1$において$f(x) \geq 0$.
  よって, $x\geq 3$において$3^x > 4x+10$.
\end{rem}
\begin{rem}
  数学的帰納法を用いず, 以下のように示すこともできる:
  $n=0$に対し, $3^0=1=2\cdot0+1$.
  $n\geq 3$に対し, $2n\geq 6$, $2^{n-1}\geq 4$, $2^n\geq 8$
  であるので
  \begin{align*}
    3^n=(1+2)^n=\sum_{i=0}^n \binom{n}{i}2^i
    &\geq \binom{n}{0}+\binom{n}{1}2+\binom{n}{1}2^{n-1}+2^{n}
    =1+2n+2^{n-1}n+2^n
    \geq 1+6+4n+8=4n+15>4n+10
    .\qedhere
  \end{align*}
\end{rem}

\begin{prop}
  \label{p:20230821}
  $t>0$とする.
  $\forall n\in\NN$,
  $(1+t)^n\geq 1+nt$.
\end{prop}
\begin{proof**}
  $(1+t)^0=1=1+0\cdot t$である.
  また, 
  $(1+t)^n=(1+t)(1+t)^{n-1}=(1+t)^{n-1}+t(1+t)^{n-1}$であることから,
  $n$に関する数学的帰納法により示せる.
\end{proof**}
\begin{proof*}
  $t>0$とする.
  $P(n)$を次の命題とする:
  \begin{align*}
    (1+t)^n\geq 1+nt
    .
  \end{align*}
  このとき,
  すべての$n\in\NN$で$P(n)$が成り立つことを,
  数学的帰納法で示す.

  \paragraph{Base Case:}
  $P(0)$が成り立つことは, 以下から明らか:
  \begin{align*}
    (1+t)^0&=1\\
     1+0\cdot t&=1.
  \end{align*}
  \paragraph{Induction Step:}
  $n\geq 1$とし,
  $P(n-1)\implies P(n)$を示す.
  仮定から$(1+t)^{n-1}> 1+t(n-1)$である.
  また, $t>0$であるので, 
  $n\geq 1$に対し$t^2(n-1)\geq 0$である.
  よって,
  \begin{align*}
    (1+t)^n=&(1+t)(1+t)^{n-1}\\
    =&(1+t)^{n-1}+t(1+t)^{n-1}\\
    \geq&
    (1+t(n-1))+t(1+t(n-1))\\
    =&1+tn+t^2(n-1)\\
    \geq&1+tn
    .\qedhere
  \end{align*}
\end{proof*}

\begin{rem}
  数学的帰納法を用いず, 以下のように示す方が一般的だと思う:
  $n=0$に対し, $(1+t)^0=1=1+t\cdot0$.
  $n\in\ZZ_{>0}$に対し, 
  \begin{align*}
    (1+t)^n=
    \sum_{i=0}^n \binom{n}{i}t^i
    &\geq \binom{n}{0}+\binom{n}{1}t
    =1+nt
    .\qedhere
  \end{align*}
\end{rem}

\begin{prop}
  \label{p:20230823}
  $t>0$とする.
  $\forall n\in\NN$,
  $(1-t)^n\geq 1-nt$.
\end{prop}
\begin{proof**}
  $(1+t)^0=1=1+0\cdot t$である.
  また, 
  $(1-t)^n=(1-t)(1-t)^{n-1}=(1-t)^{n-1}-t(1-t)^{n-1}$であることから,
  $n$に関する数学的帰納法により示せる.
\end{proof**}
\begin{proof*}
  $t>0$とする.
  $P(n)$を次の命題とする:
  \begin{align*}
    (1-t)^n\geq 1-nt
    .
  \end{align*}
  このとき,
  すべての$n\in\NN$で$P(n)$が成り立つことを,
  数学的帰納法で示す.

  \paragraph{Base Case:}
  $P(0)$が成り立つことは, 以下から明らか:
  \begin{align*}
    (1-t)^0&=1\\
     1-0\cdot t&=1.
  \end{align*}
  \paragraph{Induction Step:}
  $n\geq 1$とし,
  $P(n-1)\implies P(n)$を示す.
  仮定から$(1-t)^{n-1}> 1-t(n-1)$である.
  また, $t>0$であるので, 
  $n\geq 1$に対し$t^2(n-1)\geq 0$である.
  よって,
  \begin{align*}
    (1-t)^n=&(1-t)(1-t)^{n-1}\\
    =&(1-t)^{n-1}-t(1-t)^{n-1}\\
    \geq&
    (1-t(n-1))-t(1-t(n-1))\\
    =&1-tn+t^2(n-1)\\
    \geq&1-tn
    .\qedhere
  \end{align*}
\end{proof*}

\newest
\begin{prop}
  \label{p:20230824}
  $\forall n\in\ZZ_{>0}$,
  $n^2\leq \left(\sum_{i=1}^{n}i\right)\left(\sum_{i=1}^{n}\frac{1}{i}\right)$.
\end{prop}
\begin{proof**}
  $1^2=1\cot 1$である.
  また, 
  \begin{align*}
    \left(\sum_{i=1}^{n}i\right)\left(\sum_{i=1}^{n}\frac{1}{i}\right)
    &=
    n+\left(\sum_{i=2}^{n}\frac{n}{i}\right)
    +\left(\sum_{i=1}^{n-1}i\right)\left(\sum_{i=1}^{n}\frac{1}{i}\right)
    \\
    &\geq
    2n-1+
    \left(\sum_{i=1}^{n-1}i\right)\left(\sum_{i=1}^{n}\frac{1}{i}\right)\\
    &\geq
    2n-1+
    \left(\sum_{i=1}^{n-1}i\right)\left(\sum_{i=1}^{n-1}\frac{1}{i}\right)
  \end{align*}
  であることから,
  $n$に関する数学的帰納法により示せる.
\end{proof**}
\begin{proof*}
  $P(n)$を次の命題とする:
  \begin{align*}
    n^2\leq \left(\sum_{i=1}^{n}i\right)\left(\sum_{i=1}^{n}\frac{1}{i}\right)
    .
  \end{align*}
  このとき,
  すべての$n\in\ZZ_{>0}$で$P(n)$が成り立つことを,
  数学的帰納法で示す.

  \paragraph{Base Case:}
  $P(1)$が成り立つことは, 以下から明らか:
  \begin{align*}
    1^2=1\cdot 1.
  \end{align*}
  \paragraph{Induction Step:}
  $P(n-1)\implies P(n)$を示す.
  仮定から$(n-1)^2\leq \left(\sum_{i=1}^{n-1}i\right)\left(\sum_{i=1}^{n-1}\frac{1}{i}\right)$である.
  また, $t>0$であるので, 
  $n\geq 1$に対し$t^2(n-1)\geq 0$である.
  よって,
  \begin{align*}
    \left(\sum_{i=1}^{n}i\right)\left(\sum_{i=1}^{n}\frac{1}{i}\right)
    &=\left(n+\sum_{i=1}^{n-1}i\right)\left(\sum_{i=1}^{n}\frac{1}{i}\right)\\
    &=n\left(\sum_{i=1}^{n}\frac{1}{i}\right)+\left(\sum_{i=1}^{n-1}i\right)\left(\sum_{i=1}^{n}\frac{1}{i}\right)\\
    &=
    n\cdot1+n\left(\sum_{i=2}^{n}\frac{1}{i}\right)+
    \left(\sum_{i=1}^{n-1}i\right)\left(\sum_{i=1}^{n}\frac{1}{i}\right)\\
    &=
    n\cdot1+\left(\sum_{i=2}^{n}\frac{n}{i}\right)
    +\left(\sum_{i=1}^{n-1}i\right)\left(\sum_{i=1}^{n}\frac{1}{i}\right)
    \\
    &\geq
    n\cdot1+\left(\sum_{i=2}^{n}1\right)+
    \left(\sum_{i=1}^{n-1}i\right)\left(\sum_{i=1}^{n}\frac{1}{i}\right)\\
    &=
    n+n-1+
    \left(\sum_{i=1}^{n-1}i\right)\left(\sum_{i=1}^{n}\frac{1}{i}\right)\\
    &=
    2n-1+
    \left(\sum_{i=1}^{n-1}i\right)\left(\sum_{i=1}^{n}\frac{1}{i}\right)\\
    &\geq
    2n-1+
    \left(\sum_{i=1}^{n-1}i\right)\left(\sum_{i=1}^{n-1}\frac{1}{i}\right)\\
    &\geq
    2n-1+(n-1)^2\\
    &=2n-1+n^2-2n+1\\
    &=n^2
    .\qedhere
  \end{align*}
\end{proof*}


\section{制限された数学的帰納法 (Limited mathematical induction)}

\section{Todo}






\begin{prop}
  $a_1=7$, $a_{n}=a_{n}^3$
  $a_n\equiv 1\pmod{3^n}$
\end{prop}


\begin{prop}
  $\forall p,q,n\in\NN$,
  $\binom{p+q}{n}=\sum_{i=0}^n\binom{p}{i}\binom{q}{n-i}$.
\end{prop}



\begin{prop}
$\sum_{i=1}^{n}\frac{1}{i}\geq \frac{2n}{n+1}$.
\end{prop}


\begin{prop}
$\sum_{i=1}^{n}\frac{1}{\sqrt{i}}<2\sqrt{n}$.
\end{prop}

\begin{prop}
$n\in\NN$とする.
  $n\geq 2$ならば, $\sum_{k=1}^n\frac{1}{k^2} < 2-\frac{1}{n}$.
\end{prop}


\begin{prop}
$\frac{a^n+b^n}{2}\geq \left(\frac{a+b}{2}\right)^n$
\end{prop}

\begin{prop}
  $a_i>0$なら,
  $\left(\frac{\sum_{i=1}^n a_i}{n}\right)^m \leq \frac{\sum_{i=1}^{n}a_i^m}{n}$.
\end{prop}

\begin{prop}
$x\in\RR$と$n\in\NN$に対し, $|\sin(nx)|\leq n|\sin(x)|$.
\end{prop}

\begin{prop}
  $n\geq 2$ならば,
  $\sum_{i=1}^{n}\frac{1}{i^2}<2-\frac{1}{n}$.
\end{prop}

\begin{prop}
  $f$を凸関数とする.
  $a_i\geq 0$, $\sum_{i=1}^n a_i=1$ならば,
  $\sum_{i}^{n} a_i f(x_i)\geq \sum_{i}^{n}  f(a_i x_i)$.
\end{prop}




\begin{prop}
$n\in\NN$とする.
  $n\geq 12$ならば, $n=4x+5y$を満たす$x,y\in\NN$が存在する.
\end{prop}


\begin{prop}
  $a_n$をfibonacci数列とする.
  $a_n=\frac{\varphi^n-\psi^n}{\varphi-\psi}$,
  ただし$\varphi=\frac{1+\sqrt{5}}{2}$,
  $\psi=\frac{1-\sqrt{5}}{2}$
  とする.
\end{prop}

\begin{prop}
  $a_n=\sqrt{a_{n-1}+2}$, $a_0=1$とする.
  $a_n<a_{n+1}$.
\end{prop}



\begin{prop}
  $a_1=1$, $a_2=2$, $a_3=6$,
  $a_n=(n^3-3n^2+2n)a_{n-3}$.
  このとき, $a_n=n!$.
\end{prop}



\begin{prop}
  $a$, $b\in\ZZ$.
  $\alpha, \beta$は$x^2-ax+b=0$の2つの解.
  このとき, $n\in\NN$に対し,
  $\alpha^n+\beta^n\in \ZZ$.
\end{prop}

\begin{prop}
  $a$, $b\in\ZZ$,  
  $(2+\sqrt{3})^n=a+\sqrt{3}b$
  ならば,
  $(2-\sqrt{3})^n=a-\sqrt{3}b$.
\end{prop}

\begin{prop}
  $a$, $b\in\ZZ$,  
  $(3+2\sqrt{2})^n=a+\sqrt{2}b$
  ならば,
  $(3-2\sqrt{2})^n=a-\sqrt{2}b$.
\end{prop}


\begin{prop}
  $(3+2\sqrt{2})^n+(3-2\sqrt{2})^n\in\ZZ$.
\end{prop}

\begin{prop}
  $\frac{(5+2\sqrt{6})^n+(5-2\sqrt{6})^n}{2}\in\ZZ$.
\end{prop}

\begin{prop}
  $x+\frac{1}{x}\in\ZZ$ならば, 
  $x^n+\frac{1}{x^n}\in\ZZ$.
\end{prop}


\begin{prop}
  $x^4+y^4=z^4$となる整数は存在しない.
\end{prop}

\begin{prop}
  $p$が素数なら$\sqrt{p}$は無理数.
\end{prop}

\begin{prop}
  Fibonacci数列は互いに素.
\end{prop}



\begin{prop}
  $n\geq 4$なら$n!>2^n$
\end{prop}



\begin{prop}
  $\frac{\sum_{i=1}^n a_i}{n} \geq \sqrt[n]{\prod\sum_{i=1}^n a_i}$.
\end{prop}


\begin{prop}
$\frac{d^n}{dx^n}(fg)=\sum_{i=0}^n\binom{n}{i}\frac{d^i}{dx^i}f \cdot \frac{d^{n-i}}{dx^{n-i}}g$
\end{prop}




\begin{prop}
  $0\leq 3a_n\leq\sum_{i=0}^n a_i$をみたす
  $a_n$は$a_n=0$.
\end{prop}

\begin{prop}
  $|\sum_{i=1}^{n}a_i|\leq \frac{1}{n}$をみたす
  $a_i\in\Set{\frac{1}{i},-\frac{1}{i}}$が存在する.
\end{prop}

\begin{prop}
一筆書き.
\end{prop}

\begin{prop}
ほうじょ原理
\end{prop}

\begin{prop}
  $n\geq 4$とする.
  凸$n$角形の対角線の総数は$\frac{n(n-3)}{2}$.
\end{prop}

\begin{prop}
  一般の位置にある平面上の$n$本の直線は,
  平面を$\frac{n^2+n+2}{2}$個の領域に分ける.
\end{prop}

\begin{prop}
  素因数分解の一意性.
\end{prop}

\begin{prop}
  あまりの存在.
\end{prop}

% ひびGB
\begin{prop}
  ユークリドごじょほう.  整数, 多項式, ふち.
\end{prop}

\begin{prop}
  1変数多項式かんの単項式順序はただ一つ
\end{prop}

\begin{prop}
  Dicksonの補題
  %2.11
\end{prop}

\begin{prop}
  た変数多項式のHilbertの基底定理.
  % 2.21
\end{prop}

\begin{prop}
  まっこーれいの定理.
  % 2.3.9
\end{prop}

\begin{prop}
  GBによるた変数割り算アルゴリズムの標準形の存在.
\end{prop}


% ひび多面体
\begin{prop}
$\bigcap_{Y\colon \text{凸, $X\subset Y$}}Y=\Set{\sum t_i x_i|t_i\geq 0, \sum t_i=1}$
\end{prop}

\begin{prop}
ヒルベルトの基底定理
\end{prop}

\begin{prop}
標準次数のときの
  ネーターのせいきかていり
%せいせいげんの個数に関するきのうほう
\end{prop}

%かつらガロア論
\begin{prop}
  %1.3.1
  分解たいの存在.
\end{prop}

\begin{prop}
  %1.3.7
  延長の存在.
\end{prop}

\begin{prop}
  %1.5.16
  有限次ぶんりかくだいたいは単純
\end{prop}

\begin{prop}
  %1.6.11
  $char F=p$, $f\in F[x]$がきやくなら,
  ぶんり多項式$h$をつかって$f(x)=h(x^{p^e})$とかける.
\end{prop}

\begin{prop}
  %1.7.9
  $E/F$有限次拡大.
  $G$を$E$の$F$上の自己同型群とする.
  $E$は$F[X]$のある分離多項式$f$の最小分解体であるなら,
  $E^G=F$.
\end{prop}


% グラフ

\begin{prop}
  $G=(V,E)$がsimple graphで$V>2$.
  $\forall v$, $v$のじすうは2以上.
  このとき, $G$に閉路がある.
%へんの数
\end{prop}

\begin{prop}
  $C$, $C_1$,\ldots, $C_k$, 閉路, どの2つも辺を共有しない.
  $C$は各$C_i$と頂点を共有する.

  $C$, $C_1$,\ldots, $C_k$の全ての辺を使った閉路がある.
%k
\end{prop}


\begin{prop}
  $G=(V,E)$連結.
  $\forall v$, $v$のじすうは2の倍数.
  このとき, $G$にオイラー閉路がある.
%へんの数
\end{prop}



%%
\begin{prop}
きほんたいしょうしきは代数独立
%f(e)=0としたときのfの次数と, 変数の数に関する帰納ほう.
\end{prop}

%%                                                                             
\begin{prop}
  (Matsumoto's theorem)
最短表示同士はbraid relationで移り合う.
\end{prop}
