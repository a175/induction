% !TeX root =./x2.tex
% !TeX program = pdfpLaTeX
\section{記号}
$\NN$は非負整数全体のなす集合,
$\ZZ_{>0}$は正の実数全体のなす集合.
\section{通常の数学的帰納法によるもの}
\subsection{数学的帰納法は通常は用いないもの}
\begin{prop}
  $\forall n\in\NN$,
  $\sum_{i=0}^{n}i=\frac{n(n+1)}{2}$.
\end{prop}
\begin{proof**}
  $0=\frac{0(0+1)}{2}$である.
  また,
  $n+\sum_{i=0}^{n-1}i=\sum_{i=0}^{n}i$,
  $n+\frac{(n-1)n}{2}=\frac{n(n+1)}{2}$であるので,
  $n$に関する数学的帰納法により示せる.
\end{proof**}
\begin{proof*}
  $P(n)$を次の命題とする:
  \begin{align*}
    \sum_{i=0}^{n}i=\frac{n(n+1)}{2}.
  \end{align*}
  このとき, 全ての$n\in\NN$で$P(n)$が成り立つことを,
  数学的帰納法で示す.

  \paragraph{Base Case:}
  $P(0)$が成り立つことは, 以下から明らか:
  \begin{align*}
    \sum_{i=0}^{0}i&=0,\\
    \frac{0(0+1)}{2}&=0.
  \end{align*}

  \paragraph{Induction Step:}
  $P(n-1)\implies P(n)$を示す.
  仮定から$\sum_{i=0}^{n-1}i=\frac{(n-1)n}{2}$であるので,
  \begin{align*}
    \sum_{i=0}^{n}i&=n+\sum_{i=0}^{n-1}i\\
    &=n+\frac{(n-1)n}{2}\\
    &=\frac{2n+(n-1)n}{2}\\
    &=\frac{2n+n^2-n}{2}\\
    &=\frac{n^2+n}{2}\\
    &=\frac{n(n+1)}{2}.
    \qedhere
  \end{align*}
\end{proof*}
\begin{rem}
  数学的帰納法を用いず, 以下のように示す方が一般的だと思う:
  \begin{align*}
    \sum_{i=0}^{n}i&=\frac{\sum_{i=0}^{n}i + \sum_{i=0}^{n}i}{2}\\
    &=\frac{\sum_{i=0}^{n}i + \sum_{i=0}^{n}(n-i)}{2}\\
    &=\frac{\sum_{i=0}^{n}(i + n-i)}{2}\\
    &=\frac{\sum_{i=0}^{n}n}{2}\\
    &=\frac{n\sum_{i=0}^{n}1}{2}\\
    &=\frac{n(n+1)}{2}.
  \end{align*}
\end{rem}

\begin{prop}
  $\forall n\in\NN$,
  $\sum_{i=0}^{n}2i=n(n+1).$
\end{prop}
\begin{proof**}
  $2\cdot 0=0=0(0+1)$である.
  また,
  $2n+\sum_{i=0}^{n-1}2i=\sum_{i=0}^{n}2i$,
  $2n+(n-1)n=n(n+1)$であるので,
  $n$に関する数学的帰納法により示せる.
\end{proof**}


\begin{proof*}
  $P(n)$を次の命題とする:
  \begin{align*}
    \sum_{i=0}^{n}2i=n(n+1).
  \end{align*}
  このとき, 全ての$n\in\NN$で$P(n)$が成り立つことを,
  数学的帰納法で示す.

  \paragraph{Base Case:}
  $P(0)$が成り立つことは, 以下から明らか:
  \begin{align*}
    \sum_{i=0}^{0}2i&=2\cdot 0=0,\\
    0(0+1)&=0.
  \end{align*}

  \paragraph{Induction Step:}
  $P(n-1)\implies P(n)$を示す.
  仮定から$\sum_{i=0}^{n-1}2i=(n-1)n$であるので,
  \begin{align*}
    \sum_{i=0}^{n}2i&=2n+\sum_{i=0}^{n-1}2i\\
    &=2n+(n-1)n\\
    &=2n+n^2-n\\
    &=n^2+n\\
    &=n(n+1).
    \qedhere
  \end{align*}
\end{proof*}

\begin{rem}
  数学的帰納法を用いず, 以下のように示す方が一般的だと思う:
  \begin{align*}
    \sum_{i=0}^{n}2i&=
    2\sum_{i=0}^{n}i
    =2\frac{n(n+1)}{2}=n(n+1).
  \end{align*}
\end{rem}

\begin{prop}
  $\forall n\in\ZZ_{>0}$, $\sum_{i=1}^{n}(2i-1)=n^2$.
\end{prop}

\begin{proof**}
  $2\cdot 1-1=1=1^2$である.
  また,
  $(2n-1)+\sum_{i=1}^{n-1}(2i-1)=\sum_{i=1}^{n}(2i-1)$,
  $(2n-1)+(n-1)^2=n^2$であるので,
  $n$に関する数学的帰納法により示せる.
\end{proof**}

\begin{proof*}
  $P(n)$を次の命題とする:
  \begin{align*}
    \sum_{i=1}^{n}(2i-1)=n^2.
  \end{align*}
  このとき,
  全ての$n\in\ZZ_{>0}$で$P(n)$が成り立つことを,
  数学的帰納法で示す.

  \paragraph{Base Case:}
  $P(1)$が成り立つことは, 以下から明らか:
  \begin{align*}
    \sum_{i=1}^{1}(2i-1)&=2\cdot 0-1=1,\\
    1^2&=1.
  \end{align*}

  \paragraph{Induction Step:}
  $P(n-1)\implies P(n)$を示す.
  仮定から$\sum_{i=1}^{n-1}(2i-1)=(n-1)^2$であるので,
  \begin{align*}
    \sum_{i=1}^{n}(2i-1)&=2n-1+\sum_{i=0}^{n-1}(2i-1)\\
    &=2n-1+(n-1)^2\\
    &=2n-1+n^2-2n+1\\
    &=n^2.
    \qedhere
  \end{align*}
\end{proof*}

\begin{rem}
  数学的帰納法を用いず, 以下のように示す方が一般的だと思う:
  \begin{align*}
    \sum_{i=1}^{n}(2i-1)&=
    2\sum_{i=1}^{n}i-\sum_{i=1}^n 1
    =2\frac{n(n+1)}{2}-n=n(n+1)-n=n^2.
  \end{align*}
\end{rem}

\begin{prop}
  $\forall n\in\NN$, $\sum_{i=0}^{n}i^2=\frac{n(n+1)(2n+1)}{6}$.
\end{prop}
\begin{proof**}
  $0^2=0=\frac{0\cdot 1\cdot 1}{6}$である.
  また,
  $n^2+\sum_{i=0}^{n-1}i^2=\sum_{i=0}^{n}i^2$,
  $n^2+\frac{(n-1)n(2n-1)}{6}=\frac{2n^3-3n^2+n+6n^2}{6}=\frac{2n^3+3n^2+n}{6}=\frac{n(n+1)(2n+1)}{6}$であるので,
  $n$に関する数学的帰納法により示せる.
\end{proof**}


\begin{proof*}
  $P(n)$を次の命題とする:
  \begin{align*}
    \sum_{i=0}^{n}i^2=\frac{n(n+1)(2n+1)}{6}.
  \end{align*}
  このとき,
  全ての$n\in\NN$で$P(n)$が成り立つことを,
  数学的帰納法で示す.

  \paragraph{Base Case:}
  $P(0)$が成り立つことは, 以下から明らか:
  \begin{align*}
    \sum_{i=1}^{1}i^2&=0^2=0,\\
    \frac{0(0+1)(2\cdot 0+1)}{6}&=0.
  \end{align*}

  \paragraph{Induction Step:}
  $P(n-1)\implies P(n)$を示す.
  仮定から$\sum_{i=0}^{n-1}i^2=\frac{(n-1)n(2n-1)}{6}$であるので,
  \begin{align*}
    \sum_{i=1}^{n}i^2&=n^2+\sum_{i=0}^{n-1}i^2\\
    &=\frac{6n^2+2n^3-3n^2+n}{6}\\
    &=\frac{2n^3+3n^2+n}{6}\\
    &=\frac{n(n+1)(2n+1)}{6}
    .\qedhere
  \end{align*}
\end{proof*}

\begin{rem}
  数学的帰納法を用いず, 以下のように示す方が一般的だと思う:
  \begin{align*}
    S&=\sum_{i=0}^n i^2\\
    T&=\sum_{i=1}^{n}(i^3-(i-1)^3)
  \end{align*}
  とする.
  このとき,
  \begin{align*}
    T&=\sum_{i=1}^{n}i^3-\sum_{i=1}^n (i-1)^3\\
    &=\sum_{i=1}^{n}i^3-\sum_{i=0}^{n-1} i^3)\\
    &=(n^3+\sum_{i=1}^{n-1}i^3)-(\sum_{i=1}^{n-1} i^3)+0^3)\\
    &=n^3.
  \end{align*}
  一方次のようにも計算できる:
  \begin{align*}
    T
    &=\sum_{i=1}^{n}(i^3- (i^3-3i^2+3i-1))\\
    &=\sum_{i=1}^{n}-(-3i^2+3i-1)\\
    &=3\sum_{i=1}^{n}i^2-3\sum_{i=1}^{n}i+\sum_{i=1}^{n}1\\
    &=3S-3\frac{n(n+1)}{2}+n.
  \end{align*}
  したがって,
  \begin{align*}
    n^3&=3S-3\frac{n(n+1)}{2}+n\\
    3S&=n^3+3\frac{n(n+1)}{2}-n\\
    &=\frac{2n^3+3n(n+1)-2n}{2}\\
    &=\frac{n(2n^2+3(n+1)-2)}{2}\\
    &=\frac{n(2n^2+3n-1)}{2}\\
    &=\frac{n(n+1)(2n-1)}{2}\\
    S&=\frac{n(n+1)(2n-1)}{6}.
  \end{align*}
\end{rem}

\begin{prop}
  $\forall n\in\ZZ_{>0}$, $\sum_{i=1}^{n}(2i-1)^2=\frac{n(2n-1)(2n+1)}{3}$.
\end{prop}
\begin{proof**}
  $1^2=1=\frac{1\cdot 1\cdot 3}{3}$である.
  また,
  \begin{align*}
    (2n-1)^2+\sum_{i=1}^{n-1}(2i-1)^2&=\sum_{i=1}^{n}(2i-1)^2\\
    (2n-1)^2+\frac{(n-1)(2n-3)(2n-1)}{3}
    &=\frac{3(2n-1)^2+(n-1)(2n-3)(2n-1)}{3}\\
    &=\frac{(2n-1)(3(2n-1)+(n-1)(2n-3))}{3}\\
    &=\frac{(2n-1)(6n-3+n^2-5n+3))}{3}\\
    &=\frac{(2n-1)(2n^2+n)}{3}\\
    &=\frac{n(2n-1)(2n+1)}{3}
  \end{align*}
  であるので,
  $n$に関する数学的帰納法により示せる.
\end{proof**}
\begin{proof*}
  $P(n)$を次の命題とする:
  \begin{align*}
    \sum_{i=1}^{n}(2i-1)^2=\frac{n(2n-1)(2n+1)}{3}
  \end{align*}
  このとき,
  全ての$n\in\ZZ_{>0}$で$P(n)$が成り立つことを,
  数学的帰納法で示す.

  \paragraph{Base Case:}
  $P(1)$が成り立つことは, 以下から明らか:
  \begin{align*}
    \sum_{i=1}^{1}(2i-1)^2&=1^2=1,\\
    \frac{1(2\cdot 1-1)(2\cdot 1+1)}{3}&=1.
  \end{align*}

  \paragraph{Induction Step:}
  $P(n-1)\implies P(n)$を示す.
  仮定から$\sum_{i=1}^{n-1}(2i-1)^2=\frac{(n-1)(2n-3)(2n-1)}{3}$であるので,
  \begin{align*}
    \sum_{i=1}^{n}(2i-1)^2&=(2n-1)^2+\sum_{i=0}^{n-1}(2i-1)^2\\
    &=(2n-1)^2+\frac{(n-1)(2n-3)(2n-1)}{3}\\
    &=\frac{3(2n-1)^2+(n-1)(2n-3)(2n-1)}{3}\\
    &=\frac{(2n-1)(3(2n-1)+(n-1)(2n-3))}{3}\\
    &=\frac{(2n-1)((6n-3)+(2n^2-5n+3))}{3}\\
    &=\frac{(2n-1)(2n^2+n)}{3}\\
    &=\frac{n(2n-1)(2n+1)}{3}
    .\qedhere
  \end{align*}
\end{proof*}


\begin{rem}
  数学的帰納法を用いず, 以下のように示す方が一般的だと思う:
  \begin{align*}
    \sum_{i=1}^{n}(2i-1)^2&=
    4\sum_{i=1}^{n}i^2-4\sum_{i=1}^n i+\sum_{i=1}^n 1\\
    &=
    4\frac{n(n+1)(2n+1)}{6}
    -4\frac{n(n+1)}{2}+n\\
    &=\frac{4n(n+1)(2n+1)-12n(n+1)+6n}{6}\\
    &=\frac{n(4(n+1)(2n+1)-12(n+1)+6)}{6}\\
    &=\frac{n((n+1)(4(2n+1)-12)+6)}{6}\\
    &=\frac{n((n+1)(8n-8)+6)}{6}\\
    &=\frac{n(8(n+1)(n-1)+6)}{6}\\
    &=\frac{n(8(n^2-1)+6)}{6}\\
    &=\frac{n(8n^2-2)}{6}\\
    &=\frac{n(4n^2-1)}{3}\\
    &=\frac{n(2n-1)(2n+1)}{3}.
  \end{align*}
\end{rem}


\begin{prop}
  $\forall n\in\NN$, $\sum_{i=0}^{n}(2i)^2=\frac{2n(n+1)(2n+1)}{3}$.
\end{prop}
\begin{proof**}
  $0^2=0=\frac{2\cdot 0\cdot 1\cdot 1}{3}$である.
  また,
    $(2n)^2+\sum_{i=1}^{n-1}(2i)^2=\sum_{i=1}^{n}(2i)^2$,
    $(2n)^2+\frac{2(n-1)n(2n-1)}{3}=\frac{2n(n+1)(2n+1)}{3}$
  であるので,
  $n$に関する数学的帰納法により示せる.
\end{proof**}


\begin{proof*}
  $P(n)$を次の命題とする:
  \begin{align*}
    \sum_{i=0}^{n}(2i)^2=\frac{2n(n+1)(2n+1)}{3}
  \end{align*}
  このとき,
  全ての$n\in\NN$で$P(n)$が成り立つことを,
  数学的帰納法で示す.

  \paragraph{Base Case:}
  $P(0)$が成り立つことは, 以下から明らか:
  \begin{align*}
    \sum_{i=0}^{0}(2i)^2&=0^2=0,\\
    \frac{2\cdot 0(0+1)(2\cdot 0+1)}{3}&=0.
  \end{align*}

  \paragraph{Induction Step:}
  $P(n-1)\implies P(n)$を示す.
  仮定から$\sum_{i=0}^{n-1}(2i)^2=\frac{2(n-1)n(2n-1)}{3}$であるので,
  \begin{align*}
    \sum_{i=1}^{n}(2i)^2&=(2n)^2+\sum_{i=0}^{n-1}(2i)^2\\
    &=(2n)^2+\frac{2(n-1)n(2n-1)}{3}\\
    &=\frac{3(2n)^2+2(n-1)n(2n-1)}{3}\\
    &=\frac{2n(6n+(n-1)(2n-1))}{3}\\
    &=\frac{2n(6n+2n^2-3n+1)}{3}\\
    &=\frac{2n(2n^2+3n+1)}{3}\\
    &=\frac{2n(n+1)(2n+1)}{3}
    .\qedhere
  \end{align*}
\end{proof*}

\section{制限された数学的帰納法 (Limited mathematical induction)}


\section{Todo}






\begin{prop}
  $n\in\NN$に対し, $\sum_{i=0}^{2n}(-1)^ii^2=-n(2n+1)$.
\end{prop}

\begin{prop}
  $\left(\sum_{i=0}^n i\right)^2=\sum_{i=0}^n i^3$
\end{prop}

\begin{prop}
  $\sum_{i=0}^n i^3=\frac{n^2(n+1)^2}{4}$
\end{prop}

\begin{prop}
  $\sum_{i=0}^n i(i+1)=\frac{n(n+1)(n+2)}{3}$
\end{prop}

\begin{prop}
  $\sum_{i=0}^n \prod_{k=0}^{m-1}(i+k)=\frac{\prod_{k=0}^{m}(n+k)}{m}$
\end{prop}

\begin{prop}
  $\sum_{i=1}^n (2i-1)2i=\frac{n(n+1)(4n-1)}{3}$
\end{prop}

\begin{prop}
  $\sum_{i=1}^{2n}\frac{(-1)^i}{i}=\sum_{i=1}^n\frac{1}{n+i}$
\end{prop}

\begin{prop}
  $\sum_{i=1}^n (3i-2)2^{i-1}=(3n-5)2^n+5$
\end{prop}



\begin{prop}
  $(2n)!!=2^nn!$
\end{prop}
\begin{prop}
  $(2n+1)!!=\frac{(2n+1)!}{2^nn!}$
\end{prop}
\begin{prop}
  $\frac{2n!}{n!}=2^n(2n-1)!!$
\end{prop}

\begin{prop}
  $\sum_{i=1}^n\left( \frac{1}{2i-1}-\frac{1}{2i}\right)=\sum_{i=n+1}^{2n}\frac{1}{i}$.
\end{prop}

\begin{prop}
  $\sum_{i=1}^n\frac{1}{i(i+1)}=\frac{n}{n+1}$.
\end{prop}

\begin{prop}
  $\sum_{i=1}^n\prod_{k=0}^{m}\frac{1}{i+k}=\frac{1}{m}\left(\frac{1}{m!}-\prod_{k=1}^m\frac{1}{n+k}\right)$.
\end{prop}


\begin{prop}
  $\sum_{i=1}^n\frac{1}{(2i-1)(2i+1)}=\frac{n}{2n+1}$.
\end{prop}

\begin{prop}
  $\sum_{i=1}^n\frac{i}{2^i}=2-\frac{n+2}{2^n}$.
\end{prop}

\begin{prop}
  $\sum_{i=1}^n ix^i=\frac{nx^{n+2}-(n+1)x^{n+1}+x}{(x-1)^2}$.
\end{prop}

\begin{prop}
  $\sum_{i=1}^n \frac{1}{\sqrt{i}+\sqrt{i+1}}=\sqrt{n+1}-1$.
\end{prop}

\begin{prop}
$\sum_{i=1}^{n}\frac{1}{i}\geq \frac{2n}{n+1}$.
\end{prop}

\begin{prop}
$\left(\sum_{i=1}^{n}i\right)\left(\sum_{i=1}^{n}\frac{1}{i}\right)\geq n^2$.
\end{prop}

\begin{prop}
$\sum_{i=1}^{n}\frac{1}{\sqrt{i}}<2\sqrt{n}$.
\end{prop}

\begin{prop}
$n\in\NN$とする.
  $n\geq 2$ならば, $\sum_{k=1}^n\frac{1}{k^2} < 2-\frac{1}{n}$.
\end{prop}

\begin{prop}
  $n\geq 2$ならば,  $h>0$に対し,
$(1+t)^n<1+nt$.
\end{prop}
\begin{prop}
  $n\geq 2$ならば,  $h>0$に対し,
$(1-t)^n<1-nt$.
\end{prop}

\begin{prop}
$\frac{a^n+b^n}{2}\geq \left(\frac{a+b}{2}\right)^n$
\end{prop}

\begin{prop}
  $a_i>0$なら,
  $\left(\frac{\sum_{i=1}^n a_i}{n}\right)^m \leq \frac{\sum_{i=1}^{n}a_i^m}{n}$.
\end{prop}

\begin{prop}
$x\in\RR$と$n\in\NN$に対し, $|\sin(nx)|\leq n|\sin(x)|$.
\end{prop}

\begin{prop}
  $n\geq 2$ならば,
  $\sum_{i=1}^{n}\frac{1}{i^2}<2-\frac{1}{n}$.
\end{prop}

\begin{prop}
  $f$を凸関数とする.
  $a_i\geq 0$, $\sum_{i=1}^n a_i=1$ならば,
  $\sum_{i}^{n} a_i f(x_i)\geq \sum_{i}^{n}  f(a_i x_i)$.
\end{prop}


\begin{prop}
$n\in\NN$に対し, $2^n> n$.
\end{prop}

\begin{prop}
  $2^n>n^2$
\end{prop}

\begin{prop}
$n\in\NN$とする.
  $n\geq 2$ならば, $2^n \geq n+2$.
\end{prop}

\begin{prop}
$n\in\NN$とする.
  $n\geq 2$ならば, $3^n \geq 2n+1$.
\end{prop}


\begin{prop}
$n\in\NN$とする.
  $n\geq 3$ならば, $2^n\geq n+5$.
\end{prop}

\begin{prop}
$n\in\NN$とする.
  $n\geq 3$ならば, $2^n\geq 2n+1$.
\end{prop}

\begin{prop}
$n\in\NN$とする.
  $n\geq 3$ならば, $3^n\geq 4n+10$.
\end{prop}

\begin{prop}
$n\in\NN$とする.
  $n\geq 4$ならば, $2^n\geq 3n$.
\end{prop}


\begin{prop}
$n\in\NN$とする.
  $n\geq 5$ならば, $2^n\geq n^2$.
\end{prop}

\begin{prop}
$n\in\NN$とする.
  $n\geq 5$ならば, $2^n\geq n^2-2n+15$.
\end{prop}


\begin{prop}
$n\in\NN$とする.
  $n\geq 10$ならば, $2^n\geq 10n^2$.
\end{prop}

\begin{prop}
$n\in\NN$とする.
  $n\geq 10$ならば, $2^n\geq n^3$.
\end{prop}

\begin{prop}
$\prod_{i=1}^n i \equiv 0 \pmod{n!}$.
\end{prop}



\begin{prop}
$n\in\NN$とする.
  $n\geq 12$ならば, $n=4x+5y$を満たす$x,y\in\NN$が存在する.
\end{prop}


\begin{prop}
  $a_n$をfibonacci数列とする.
  $a_n=\frac{\varphi^n-\psi^n}{\varphi-\psi}$,
  ただし$\varphi=\frac{1+\sqrt{5}}{2}$,
  $\psi=\frac{1-\sqrt{5}}{2}$
  とする.
\end{prop}

\begin{prop}
  $a_n=\sqrt{a_{n-1}+2}$, $a_0=1$とする.
  $a_n<a_{n+1}$.
\end{prop}


\begin{prop}
  $7^n-2n-1\equiv 0\pmod{4}$
\end{prop}

\begin{prop}
  $1000^n+(-1)^{n-1}\equiv 0\pmod{7}$
\end{prop}

\begin{prop}
  $3^{3n}-2^n\equiv 0\pmod{25}$
\end{prop}

\begin{prop}
  $3^{n}-2^n+3\equiv 0\pmod{4}$
\end{prop}

\begin{prop}
  $3^{3n}+7^{2n-1}\equiv 0\pmod{11}$
\end{prop}

\begin{prop}
  $4\cdot 3^{2n-1}+2^{4n}\equiv 0\pmod{28}$
\end{prop}

\begin{prop}
  $a_1=7$, $a_{n}=a_{n}^3$
  $a_n\equiv 1\pmod{3^n}$
\end{prop}

\begin{prop}
  $a_1=1$, $a_2=2$, $a_3=6$,
  $a_n=(n^3-3n^2+2n)a_{n-3}$.
  このとき, $a_n=n!$.
\end{prop}



\begin{prop}
  $a$, $b\in\ZZ$.
  $\alpha, \beta$は$x^2-ax+b=0$の2つの解.
  このとき, $n\in\NN$に対し,
  $\alpha^n+\beta^n\in \ZZ$.
\end{prop}

\begin{prop}
  $a$, $b\in\ZZ$,  
  $(2+\sqrt{3})^n=a+\sqrt{3}b$
  ならば,
  $(2-\sqrt{3})^n=a-\sqrt{3}b$.
\end{prop}

\begin{prop}
  $a$, $b\in\ZZ$,  
  $(3+2\sqrt{2})^n=a+\sqrt{2}b$
  ならば,
  $(3-2\sqrt{2})^n=a-\sqrt{2}b$.
\end{prop}


\begin{prop}
  $(3+2\sqrt{2})^n+(3-2\sqrt{2})^n\in\ZZ$.
\end{prop}

\begin{prop}
  $\frac{(5+2\sqrt{6})^n+(5-2\sqrt{6})^n}{2}\in\ZZ$.
\end{prop}

\begin{prop}
  $x+\frac{1}{x}\in\ZZ$ならば, 
  $x^n+\frac{1}{x^n}\in\ZZ$.
\end{prop}


\begin{prop}
  $x^4+y^4=z^4$となる整数は存在しない.
\end{prop}

\begin{prop}
  $p$が素数なら$\sqrt{p}$は無理数.
\end{prop}

\begin{prop}
  Fibonacci数列は互いに素.
\end{prop}



\begin{prop}
  $n\geq 4$なら$n!>2^n$
\end{prop}



\begin{prop}
  $\frac{\sum_{i=1}^n a_i}{n} \geq \sqrt[n]{\prod\sum_{i=1}^n a_i}$.
\end{prop}


\begin{prop}
$\frac{d^n}{dx^n}(fg)=\sum_{i=0}^n\binom{n}{i}\frac{d^i}{dx^i}f \cdot \frac{d^{n-i}}{dx^{n-i}}g$
\end{prop}


\begin{prop}
$\binom{p+q}{n}=\sum_{i=0}^n\binom{p}{i}\binom{q}{n-i}$
\end{prop}


\begin{prop}
  $0\leq 3a_n\leq\sum_{i=0}^n a_i$をみたす
  $a_n$は$a_n=0$.
\end{prop}

\begin{prop}
  $|\sum_{i=1}^{n}a_i|\leq \frac{1}{n}$をみたす
  $a_i\in\Set{\frac{1}{i},-\frac{1}{i}}$が存在する.
\end{prop}

\begin{prop}
一筆書き.
\end{prop}

\begin{prop}
ほうじょ原理
\end{prop}

\begin{prop}
  $n\geq 4$とする.
  凸$n$角形の対角線の総数は$\frac{n(n-3)}{2}$.
\end{prop}

\begin{prop}
  一般の位置にある平面上の$n$本の直線は,
  平面を$\frac{n^2+n+2}{2}$個の領域に分ける.
\end{prop}

\begin{prop}
  素因数分解の一意性.
\end{prop}

\begin{prop}
  あまりの存在.
\end{prop}

% ひびGB
\begin{prop}
  ユークリドごじょほう.  整数, 多項式, ふち.
\end{prop}

\begin{prop}
  1変数多項式かんの単項式順序はただ一つ
\end{prop}

\begin{prop}
  Dicksonの補題
  %2.11
\end{prop}

\begin{prop}
  た変数多項式のHilbertの基底定理.
  % 2.21
\end{prop}

\begin{prop}
  まっこーれいの定理.
  % 2.3.9
\end{prop}

% ひび多面体
\begin{prop}
$\bigcap_{Y\colon \text{凸, $X\subset Y$}}Y=\Set{\sum t_i x_i|t_i\geq 0, \sum t_i=1}$
\end{prop}

\begin{prop}
ヒルベルトの基底定理
\end{prop}

\begin{prop}
ネーターのせいきかていり
\end{prop}

%かつらガロア論
\begin{prop}
  %1.3.1
  分解たいの存在.
\end{prop}

\begin{prop}
  %1.3.7
  延長の存在.
\end{prop}

\begin{prop}
  %1.5.16
  有限次ぶんりかくだいたいは単純
\end{prop}

\begin{prop}
  %1.6.11
  $char F=p$, $f\in F[x]$がきやくなら,
  ぶんり多項式$h$をつかって$f(x)=h(x^{p^e})$とかける.
\end{prop}


% グラフ

\begin{prop}
  $G=(V,E)$がsimple graphで$V>2$.
  $\forall v$, $v$のじすうは2以上.
  このとき, $G$に閉路がある.
%へんの数
\end{prop}

\begin{prop}
  $C$, $C_1$,\ldots, $C_k$, 閉路, どの2つも辺を共有しない.
  $C$は各$C_i$と頂点を共有する.

  $C$, $C_1$,\ldots, $C_k$の全ての辺を使った閉路がある.
%k
\end{prop}


\begin{prop}
  $G=(V,E)$連結.
  $\forall v$, $v$のじすうは2の倍数.
  このとき, $G$にオイラー閉路がある.
%へんの数
\end{prop}

